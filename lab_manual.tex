%%%%%%%%%%%%%%%%%%%%%%%%%%%%%%%%%%%%%%%%%
% Tufte-Style Book (Minimal Template)
% LaTeX Template
% Version 1.0 (5/1/13)
%
% This template has been downloaded from:
% http://www.LaTeXTemplates.com
%
% License:
% CC BY-NC-SA 3.0 (http://creativecommons.org/licenses/by-nc-sa/3.0/)
%
% IMPORTANT NOTE:
% In addition to running BibTeX to compile the reference list from the .bib
% file, you will need to run MakeIndex to compile the index at the end of the
% document.
%
%%%%%%%%%%%%%%%%%%%%%%%%%%%%%%%%%%%%%%%%%

%----------------------------------------------------------------------------------------
%	PACKAGES AND OTHER DOCUMENT CONFIGURATIONS
%----------------------------------------------------------------------------------------

\PassOptionsToPackage{nobottomtitles}{titlesec}
\documentclass{tufte-book} % Use the tufte-book class which in turn
                           % uses the tufte-common class

\definecolor{dartmouthgreen}{RGB}{0, 105, 62}

\hypersetup{colorlinks=true,linkcolor=dartmouthgreen} % Comment this line if you don't wish to have colored links

\usepackage{microtype} % Improves character and word spacing

%\usepackage{lipsum} % Inserts dummy text

\usepackage{booktabs} % Better horizontal rules in tables

\usepackage{graphicx} % Needed to insert images into the document
\graphicspath{{graphics/}} % Sets the default location of pictures
\setkeys{Gin}{width=\linewidth,totalheight=\textheight,keepaspectratio}
% Improves figure scaling

\usepackage[export]{adjustbox}

\usepackage{fancyvrb} % Allows customization of verbatim environments
\fvset{fontsize=\normalsize} % The font size of all verbatim text can be changed here

\newcommand{\hangp}[1]{\makebox[0pt][r]{(}#1\makebox[0pt][l]{)}} % New command to create parentheses around text in tables which take up no horizontal space - this improves column spacing
\newcommand{\hangstar}{\makebox[0pt][l]{*}} % New command to create asterisks in tables which take up no horizontal space - this improves column spacing

\usepackage{xspace} % Used for printing a trailing space better than
                    % using a tilde (~) using the \xspace command

\usepackage{hyperref} %web links/URLs

\usepackage{enumitem,amssymb}
\newlist{todolist}{itemize}{2}
\setlist[todolist]{label=$\square$}

\newcommand{\monthyear}{\ifcase\month\or January\or February\or March\or April\or May\or June\or July\or August\or September\or October\or November\or December\fi,\space\number\year} % A command to print the current month and year

\newcommand{\openepigraph}[2]{ % This block sets up a command for printing an epigraph with 2 arguments - the quote and the author
\begin{fullwidth}
\sffamily%\large
\begin{doublespace}
\noindent\allcaps{#1}\\ % The quote
\noindent\allcaps{#2} % The author
\end{doublespace}
\end{fullwidth}
}

\newcommand{\ourschool}{Dartmouth College}

\newcommand{\blankpage}{\newpage\hbox{}\thispagestyle{empty}\newpage} % Command to insert a blank page

\usepackage{makeidx} % Used to generate the index
\makeindex % Generate the index which is printed at the end of the document

%----------------------------------------------------------------------------------------
%	BOOK META-INFORMATION
%----------------------------------------------------------------------------------------

\title{Lab Manual} % Title of the book

\author{Jeremy R. Manning, Ph.D.} % Author

\publisher{Contextual Dynamics Lab, \ourschool} % Publisher

%----------------------------------------------------------------------------------------

\begin{document}
\frontmatter

%----------------------------------------------------------------------------------------
%	EPIGRAPH
%----------------------------------------------------------------------------------------

\thispagestyle{empty}
%\openepigraph{Quotation 1}{Author, {\itshape Source}}
%\vfill
%\openepigraph{Quotation 2}{Author}
%\vfill
%\openepigraph{Quotation 3}{Author}

%----------------------------------------------------------------------------------------

\maketitle % Print the title page
%----------------------------------------------------------------------------------------
%	COPYRIGHT PAGE
%----------------------------------------------------------------------------------------

\newpage
\begin{fullwidth}
~\vfill
\thispagestyle{empty}
\setlength{\parindent}{0pt}
\setlength{\parskip}{\baselineskip}

\includegraphics[width=0.3in,left]{./lab_logo/CDL_Avatar_Cropped.png}\\\vspace{0.2in}

Copyright \copyright\ \the\year\ \thanklessauthor

\par\smallcaps{Published by the \thanklesspublisher}

\par\smallcaps{\url{http://www.context-lab.com}}

%\par License information.\index{license}

\par\textit{Current as of \monthyear}
\end{fullwidth}

%----------------------------------------------------------------------------------------

\setcounter{tocdepth}{1}
\tableofcontents % Print the table of contents

%----------------------------------------------------------------------------------------

%\listoffigures % Print a list of figures

%----------------------------------------------------------------------------------------

%\listoftables % Print a list of tables

%----------------------------------------------------------------------------------------
%	DEDICATION PAGE
%----------------------------------------------------------------------------------------

% \cleardoublepage
% ~\vfill
% \begin{doublespace}
% \noindent\fontsize{18}{22}\selectfont\itshape
% \nohyphenation
% Dedicated to my family and friends.
% \end{doublespace}
% \vfill
% \vfill

%----------------------------------------------------------------------------------------
%	INTRODUCTION
%----------------------------------------------------------------------------------------

\newcommand{\director}{Jeremy}
\newcommand{\coordinator}{\director}
\newcommand{\labmeetingtime}{Tuesdays at 12:30pm}

\cleardoublepage
\chapter{Introduction}\label{ch:intro} % Adding an asterisk leaves out this chapter from the table of contents
This lab manual is intended to provide a crash course in doing
research in the Contextual Dynamics Lab.  It describes your rights and
responsibilities as a member of the lab.  The manual also introduces
our general research approach and lab policies.

\newthought{Who is this lab manual for?}

\noindent \marginnote{\texttt{TASK:} Upon reading through this lab
  manual for the first time, please update the document to include
  your name in the \hyperref[sec:curr_members]{Lab members}
  section. Importantly, be sure to fork the
  \href{https://github.com/ContextLab/lab-manual}{GitHub repository},
  make your edit on your personal fork, and submit a pull request with
  your update. Be sure to recompile the \texttt{.tex} file after you
  make your changes so that the \texttt{.pdf} file is updated and the
  \LaTeX~compiler catches any errors you may have made. Also, feel
  free to make any additional changes you think would benefit other
  current or future lab members.  You could correct a typo, clarify
  something that’s unclear, add a comment or reference to a useful
  tool, etc.  If you'd like to get feedback on your idea first, create
  a \href{https://github.com/ContextLab/lab-manual/issues}{GitHub
    issue} describing your proposed change; you can use the existing
  issues to help decide what to focus on for your own edits if you'd
  like.}  Every new lab member should read the latest version of this
lab manual in detail and reference it later as needed.  Periodically
throughout the document, you will see margin notes with listed
\texttt{TASK} items.  Completing your read through entails: (a)
reading the contents of the manual, (b) asking current lab members
about any confusing aspects, and (c) completing the relevant
\texttt{TASK} items.  You will also see non-task \texttt{NOTE} items;
these provide helpful tips and additional commentary on the nearby
text.

This lab manual is meant to be a ``living document.'' All lab members
are welcome (and encouraged!) to submit edits that improve the
content, clarity, and overall helpfulness of this document at any
point throughout their tenure in the lab.

\newthought{What should you do if you don't understand something?}

\noindent \marginnote{\texttt{TASK:} If you haven't used \LaTeX~before
  (i.e., the document formatting language in which this manual is
  written), you'll want to
  \href{https://www.latex-project.org/get/}{download \LaTeX} and take
  a look at
  \href{https://www.latex-tutorial.com/tutorials/quick-start/}{this
    ``quick start'' tutorial}.} If you don't understand something you
read in this manual, it is important that you \textit{ask another lab
  member for help}.  Every member of the lab brings their own unique
knowledge base, training, life experiences, and perspectives.
Respecting and celebrating those differences drives the science we do.
If you're new to the lab or new to a particular technique, you might
feel like a newbie today---but chances are good that if you stick
around for a bit someone else will be seeking your expert opinion
before you know it.  In addition to learning, there's another good
reason for asking for help: if you don't understand something, there's
a reasonable chance that you've discovered a mistake or a logical
inconsistency!

\marginnote{\texttt{TASK:} When you are done reading this manual and
  carrying out all required tasks, please fill out the signature page,
  sign it (electronically), and email a PDF (of just the signature
  page) to
  \href{mailto:contextualdynamics@gmail.com}{contextualdynamics@gmail.com}.
  \textbf{You are officially a lab member once you have completed all
    tasks in this manual and receipt of your signed and filled-out
    signature page and checklist has been acknowledged by \director.}}

\newthought{Why is it worth my time to read through the manual?}

Aside from pursuing your own curiosity, a major reason that you've
decided to join an academic research lab is probably because you want
to gain training or career-advancing experiences.  This manual briefly
summarizes the collective wisdom of past and present lab members in a
way that we think will best allow you to achieve your objectives.
\textit{Learn from it}, \textit{challenge it}, and \textit{add to it}.

\newthought{What ``isn't'' this lab manual?}

\noindent This lab manual is \textit{not} intended to provide a
comprehensive overview of everything you need to know to do your
research projects.  As described next, you may not even \textit{know}
what you need to know to do your projects!  Nevertheless, you need
somewhere to start, and this is that place.

We also maintain a repository of
\href{https://github.com/ContextLab/CDL-tutorials}{lab tutorials} that
provide guidance on specific tasks.  If you are looking for help on a
particular task (or understanding a particular concept) that isn't
covered by the existing set of tutorials, please consider contributing
a tutorial of your own once you've figured things out!


\chapter{Bill of rights and responsibilities}\label{ch:billofrights}
\marginnote{\texttt{TASK:} Read
  \href{https://www.sciencemag.org/careers/2018/11/what-can-we-learn-dartmouth}{this
    letter} about defining and characterizing boundaries between lab
  members and noticing unhealthy norms.}  As a member of the
Contextual Dynamics Lab, you are entitled to certain rights, and you
agree to take on certain responsibilities.

\newthought{Your rights as a lab member}

\begin{enumerate}
\item You are entitled to a safe work environment free from
  harassment, abuse, violence, and discrimination in any form.
  \item You are entitled to be supported and respected by all lab
    members.
  \item You are entitled to openly share your scientific ideas and
    constructive feedback with all lab members.
  \item You are entitled to appropriate credit (e.g.\ authorship,
    acknowledgement, letter of recommendation) for your work and
    ideas.
\end{enumerate}

\newthought{Your responsibilities as a lab member}
\begin{enumerate}
  \item You agree to contribute to a safe work environment and to refrain
    from behaviors that harass, abuse, expose to violence, or
    discriminate.
  \item You agree to support and respect all lab members, including
    yourself.
  \item You agree to openly share your scientific ideas and constructive
    feedback with other lab members.
  \item You agree to clearly communicate and document your
    contributions to each research project (e.g.\ through GitHub
    commits and issues, reports, updates on Slack, etc.).
  \item You agree to establish open lines of communication between
    yourself and other lab members, and to address concerns or issues
    promptly and directly with the relevant parties (to the extent
    that you feel safe doing so).
  \item You agree to carry out your work with integrity and diligence,
    adhering to the highest possible standards of scientific
    excellence.
  \item You agree to utilize lab resources (including equipment,
    money, time, etc.) responsibly and sustainably.
  \item You agree to maintain a clean workspace free from clutter,
    including both personal spaces (e.g.\  desks) and shared
    areas (couch, sink, testing rooms, etc.).
\end{enumerate}

\newthought{Recourse}

\noindent If you feel your rights as a lab member have been, or are in
danger of being, violated, it is your duty to report those violations
immediately to a senior staff member (e.g.\ \director, Department
Chair, Deans, police, Title IX coordinator, ombudsman, etc.).
Similarly, if you notice others endangering others' rights, or
neglecting their responsibilities, it is your duty to report those
violations to a senior staff member.

\chapter{Official lab practices and policies}\label{ch:policy}
Our lab's practices and policies are intended to provide a framework
for \textit{maximizing efficiency}.  Achieving our peak efficiency as
a lab means we are being as scientifically productive as possible, in
terms of knowledge discovery (learning new stuff) and dissemination
(papers, talks, conference presentations, publicly released datasets,
software, etc.). It also means that our fellow lab members are
achieving their training and career objectives.  To achieve peak
efficiency, we need to succeed on three fronts:
\begin{itemize}
\item \textbf{Communication.}  We want to foster an environment where
  everyone feels comfortable contributing to the collective dialogue.
  Our lab meets regularly to discuss logistical (e.g.\ scheduling, financial,
  sociological) and technical issues.  We also use a variety of
  software packages to synchronize and facilitate communication within
  our lab and between our lab and the broader scientific community.
\item \textbf{Resource allocation.}  \marginnote{\texttt{NOTE:}
    resources, links, and discussions related to grants and other
    funding opportunities may be found in the \#grants Slack channel.
  All senior lab personel (and any interested junior personel) should join this channel to participate in
  discussions pertaining to lab resources.} Our lab resources (e.g.\
  equipment, time, money, attention) are finite.  We want to foster an
  environment where lab resources are used as efficiently as possible
  to achieve our collective goals.  We also want to foster
  sustainable use of resources by regularly pursuing research funding opportunities.
\item \textbf{Adaptability.}  The whole point of \textit{research} is that we
  don't already know the answers to the questions we're exploring or
  how to create the tools we're working on.  That means that we won't
  necessarily be able to plan out everything in advance.  We often need to
  be focused and efficient \textit{without knowing the end goal}!
\end{itemize}
Your job as a contributing lab member is to help us to achieve our
collective peak efficiency (as a lab) while also maximizing your own
training and career potential.  To do this, the Contextual Dynamics
Lab practices \textbf{agile research}, as described in the next section.

\newthought{Doing agile research}

\noindent The agile approach to research we use in the Contextual
Dynamics Lab is inspired by the
\href{http://scrumtrainingseries.com/}{Agile Movement} in the software
development world.  The idea is to create learning, adaptable teams to
work on very small bite-sized tasks.  Specifically, project teams are
designed to respond to unpredictability in research through
incremental, iterative work ``sprints.''  Each sprint lasts
approximately 1--2 weeks, and results in a demonstrable research
product (e.g.\ a draft of a paper, a draft of a grant, a completed
analysis or figure, a poster, a software tool, etc.).

This is different from traditional approaches that you may have
encountered in other labs or work environments, where a research team
might try to plan out every part of a project in advance in a series
of small steps.  We still try to break projects into tiny bite-sized
chunks, but the key insight of the agile approach is that we only need
to know what the \textit{next} chunk is, rather than attempting to
forecast out over an extended timeline.  Although it's often helpful
to have a general (if vague) sense of where things are going, we never
actually need to know where a project will ultimately end up.  The
goals and process are constantly evolving.  Perhaps the best
justification for this approach is that \textbf{the first day of a new
  research project is when you're the most clueless about what you'll
  find}.  So how could that possibly be the ideal time to plan out the
entire project?\marginnote{\texttt{NOTE:} Our adapted approach also
  draws inspiration from
  \href{http://technocalifornia.blogspot.com/2008/06/agile-research.html}{this
    blog}.}

Our \href{http://agilemanifesto.org/}{agile research manifesto} has three key
tenets:
\begin{enumerate}
\item Value \textbf{individuals and interactions} over
  \textit{processes and tools}.  To be clear, processes and tools are
  important.  But we must always keep the user or consumer in mind.
  In practice, this means that a simpler (but potentially less
  comprehensive) tool or approach may be preferable in that it could
  be easier for a reader or user to make sense of.
\item Value \textbf{working, intuitive tools and research products}
  over \textit{comprehensive documentation}.  Documentation is
  important!  But if our research products are designed in an
  intuitive way, they can (in some sense) serve as their own
  documentation.  An intuitive tool or research product with decent
  documentation is always preferable to an unintuitive tool or
  research product with comprehensive documentation.
\item Value \textbf{responding to change} over \textit{following a
    plan}.  Each new step of the research process brings new insights
  and potentially uncovers mistakes or inefficiencies.  Those
  discoveries may imply that a new direction is better than a
  previously planned one.  These are opportunities that should be
  leveraged and embraced as part of the scientific process.
\end{enumerate}
While there is value to the \textit{italicized} items, we value the
\textbf{bolded} items more.  There are
\href{http://www.agilemanifesto.org/principles.html}{twelve
  principles} we use to achieve these tenets:
\begin{enumerate}
\item Our highest priority is to benefit the research community
  through the early and continuous delivery of scientific outputs
  (ideas, presentations, papers, tutorials, tools, devices, etc.).

\item We welcome changing goals and requirements, even late in the
  process.  Doing awesome science means keeping an open mind.  Your
  original goals and plans may no longer apply as your project
  progresses.  Your original hypotheses may be proven false.  Your
  assumptions may be incompatible with your data.  Learn from these
  challenges and grow with them.  Avoid getting ``stuck'' by refusing
  to change, and allow your questions to follow where your data lead,
  rather than be constrained by your initial ideas.

\item We deliver research products frequently, in intervals ranging
  from a couple of weeks to a couple of months, with a preference for
  shorter timescales.  Before you have a concrete manifestation of
  your work (a figure, a statistic, a presentation, a paper draft, a
  dataset, a GitHub commit, etc.) you have nothing you can show the
  world for your efforts.  Produce research products, even if they're
  small and seemingly insignificant, as often as possible.  You can
  always improve on an already produced research product.

\item The product itself (software, paper, poster, presentation,
  grant) is the primary measure of progress.  Before you've
  incorporated your latest efforts into a shareable or communicable
  research product, it (effectively) doesn't exist.

\item Continuous attention to technical excellence and good design
  enhances agility.  Getting research products out regularly requires
  avoiding the temptation of aiming for perfection.  Nevertheless,
  there are often several almost-as-efficient ways to accomplishing
  tasks that vary in their design quality.  For example, consider
  whether the solution to a problem you're working on might also apply
  to other similar problems in the lab (that you or others are working
  on or have discussed).  Can you make your solution general enough to
  cover those cases?  Or, after completing a draft of a research
  product, you will likely have some insights into alternative
  (potentially better) approaches.  Can you tweak the product to
  leverage those insights?

\item Simplicity---the art of maximizing the amount of work not
  done---is essential.  Keep in mind the scope of your task.  What's
  the minimum viable set of accomplishments that will allow you to
  complete that task?  Get those done first and ``release'' your
  product (e.g.\ commit to GitHub, share via Slack, etc.).  You can
  always define a new set of goals for your next task centered around
  extending your just-released research product.  This will help to
  avoid aimless drift, whereby you spend large amounts of time on
  tasks that are, in retrospect, tangential to the main scope of work.

\item Aim to get some amount of work done \textit{every single day} on
  your project.  Commit your changes to GitHub, or document your
  progress in Slack or a Google Doc.  Maintain careful records and
  logs so that someone can pick up your work in the future (and that
  future someone might be you!).  Remember that your greatest
  collaborator is your past self, but they don't respond to emails or
  Slack messages!  Help ``future you'' maintain peak efficiency
  through methodical and well-documented work.  (Note: don't spend
  \textit{too} much time on documentation; e.g. GitHub commits are
  themselves often sufficient for documentation, since one can always
  compare different versions of a particular file.)

\item Build projects around motivated individuals.  Give them the
  environment and support they need, and trust them to get the job
  done.  Each day, ask yourself: ``am I motivated to do my best work
  on my project today?''  If the answer is ``no,'' try to understand
  why.  Is it lack of resources?  Lack of support?  Distractions?
  Ambiguous goals or research directions?  Talk to your fellow lab
  members and see how they'd approach the challenges you're facing.

\item The best architectures, requirements, and designs emerge from
  self-organizing teams.  Have you been chatting with a fellow lab
  member and you're excited about what they're working on?  Or do you
  have ideas for building on that work?  Or has a new potential team
  project emerged from a spontaneous conversation?  Think about how
  you can leverage these opportunities into research products that
  you're excited to work on!

\item The most effective and efficient method of conveying information
  within a research team is face-to-face conversation.  We use the
  \hyperref[sec: scheduling]{CDL Google calendar} to coordinate formal
  meetings between lab members.  You can also sign up for meetings
  with \director~via
  \href{https://context-lab.youcanbook.me/}{YouCanBook.Me}.  We also
  use Slack to coordinate, share notes/data, etc.  But the
  \textit{ideal} form of communication in the lab is face-to-face, and
  it often involves a whiteboard.

\item Agile research is sustainable research.  Researchers should be
  able to maintain a constant pace indefinitely.  To be sure, we
  sometimes have crunch times where we absolutely must meet a deadline
  (e.g.\ a grant submission, project milestone, etc.).  However, it is
  far more efficient to make steady progress over an extended
  timeframe than to fluctuate between periods of high and low
  productivity.  By distributing your workload you'll help yourself
  avoid burnout, preserve your mental and physical health, and allow
  yourself time to ``step back'' and think about the big picture
  (effectively getting stuff done between your work sessions!).
  Sustainable work habits also promote good communication and
  coordination between project team members.

\item At regular intervals, the team reflects on how to become more
  effective, then tunes and adjusts its behavior accordingly.  At
  minimum, all active lab members need to reflect on their projects
  once each week in your \textit{Weekly Snippet} (defined below).  In
  addition, you should
  \href{https://context-lab.youcanbook.me/}{schedule} a 30 minute
  meeting with \director~at the beginning and end of each term to
  discuss:
  \begin{itemize}
  \item What your goals are for the upcoming term (start-of-term
    meetings), or what you worked on over the prior term and how well
    you feel you met your goals (end-of-term meetings).
  \item Reflections on how things are going in the lab (you will
    report how things have been going from your perspective, and will
    receive feedback about how things have been going from \director's
    perspective).
  \item Insights you've come to regarding how we can (for your
    project, as a lab, as a department, etc.) further optimize
    efficiency or otherwise improve our work environment.
  \item Any other issues, comments, questions, or concerns that you'd
    like to check in about.
  \end{itemize}
\end{enumerate}

\newthought{Papers}

\noindent Research papers are the primary research output of our lab.
Publications are the ``currency of academia,'' in that they are
central to career advancement.  With each allocation of lab resources
(equipment, money, time) we should be asking ourselves how this
contributes to a paper.

\subsection{General procedure}
All lab papers should be coordinated with \director.  A paper starts
with a discussion of:
\begin{enumerate}
\item What the paper is going to be about
\item What the key results are
\item What the overall ``story'' is
\item The current status of various components of the project
  (e.g.\ data collection, analyses, figures, interpretation,
  literature review, etc.)
\item Who is a potential candidate for authorship on the paper
\end{enumerate}

We draft papers in \LaTeX, either on GitHub or on
\href{https://www.overleaf.com/}{Overleaf} (an online platform that
supports text editing and compiling PDFs in the browser).  Progress
should be shared regularly via Slack.

\subsection{Authorship guidelines}
\marginnote{\texttt{TASK:} Review the
  \href{https://oir.nih.gov/sites/default/files/uploads/sourcebook/documents/ethical_conduct/guidelines-authorship_contributions.pdf}{NIH
    Guidelines for Authorship}.}  The Contextual Dynamics Lab follows
the
\href{https://oir.nih.gov/sites/default/files/uploads/sourcebook/documents/ethical_conduct/guidelines-authorship_contributions.pdf}{NIH
  Guidelines for Authorship} in considering whether your contribution
to a project merits authorship on the paper.  If you have made a
non-trivial contribution to a project but did not meet the
requirements for authorship, you will instead receive a citation in
the acknowledgements section of the paper.  In general, you likely
meet the requirements for authorship if you contributed in any of the
following ways:
\begin{enumerate}
\item Drafted the manuscript (this warrants first authorship)
\item Came up with the idea or made other substantial intellectual
  contributions that meaningfully shaped the trajectory of the project
\item Carried out an original experimental study (e.g.\ that you
  designed or implemented)
\item Carried out non-trivial data analyses (e.g.\ more complicated
  than $t$-tests)
\item Contributed novel tools or resources to the project that haven't
  been published yet
\end{enumerate}

\marginnote{\texttt{NOTE:} Conference posters and abstracts generally
  have substantially less stringent authorship requirements than
  formal papers.  The general rule of thumb for posters is that all
  project team members should be co-authors.}  You are unlikely to
meet the requirements for authorship if your contributions were
limited to the following:
\begin{enumerate}
  \item Running experimental participants for an already-designed and
    coded-up study
  \item Running trivial data analyses (e.g.\ $t$-tests or similar)
  \item Getting trained by one of the other project members on a
    project-related task
  \item Training another project member on a project-related task
  \item Sharing already-published tools or resources
  \item Editing or commenting on a draft of the manuscript
\end{enumerate}

The final determination for who will be an author on each lab paper
(and in what order) will be made by \director, following open
discussions with project team members.

\newthought{Making mistakes}

\noindent The work we do is complicated, and mistakes happen.  When
you notice a mistake (a bug, misinterpretation, mislabeling, or any
other error), it is critical that you report the mistake immediately.
Whereas mistakes are unavoidable in science, negative impacts can be
minimized by fostering a workplace where reporting mistakes is
celebrated and accepted as part of the natural course of getting
things done.  Mistakes are opportunities to learn and grow, and
identifying or noticing mistakes should be celebrated as part of our
growth as scientists.  However, real harm can come from failing to
report mistakes soon enough.  There is a proverb that says ``the best
time to plant a tree was 20 years ago; the second best time is now.''
Analogously, the best time to identify and correct a mistake may have
been in the past-- but the second best time is right now!

Example scenarios (not an exhaustive list):
\begin{enumerate}
  \item You've shared a figure, statistic, or other result, and
    you've realized there's a bug in your code.
  \item You tried to collect some data and the experiment crashed or
    yielded corrupted data.
  \item You're re-reading a paper that you shared, and you notice a
    mistake or typo.
  \item You made a plan with your project team and you realized it's
    flawed in some way, or that there's potentially a better solution
    or approach.
  \item You released a software package and you've found a bug or error.
\end{enumerate}

Appropriate actions for each of the above scenarios (this should
happen immediately after you notice the mistake):
\begin{enumerate}
\item Double check, to the best of your ability, that the mistake is
  real.  This may involve checking over code, rebooting a computer and
  restarting an experiment, re-reading reference text, etc.
\item Create a GitHub issue describing the problem.  Provide
  information about how to reproduce the problem (if applicable), the
  expected behavior, and the observed behavior.  Also, provide any
  relevant system or environment information that may be necessary for
  reproducing the problem (e.g. details of the computing environment).
\item Coordinate over Slack with your project team to formulate an
  action plan.
\item Ask other lab members for help if the course of action isn't
  clear.  Also, try Google and/or Stack Exchange.
\end{enumerate}

\noindent \textit{If you think you might have caught a mistake but aren't
  sure, consult with another lab member! It never hurts to be safe!}



\newthought{Project roles}

\noindent Every project has four possible roles.  You will play one or more of
these roles on your project:
\begin{enumerate}
\item \textbf{Project Owner.}  This is the person responsible for
  maximizing ``return on investment'' of the project effort.  The project owner:
\begin{enumerate}
\item Is responsible for project vision
\item Constantly re-prioritizes the research backlog, adjusting any
  long-term expectations such as publication and release plans
\item Acts as the final arbiter of requirements questions
\item Accepts or rejects each project increment
\item Decides whether to publish/ship the project
\item Decides whether to continue development
\item Considers interests of funding bodies (e.g. NIH, NSF, DARPA,
  private organizations) and the scientific community
\item May contribute as a team member
\item Has a leadership role
\item Will usually be \director
\end{enumerate}

\item \textbf{Team Member.}  Team members are responsible for carrying
  out the project work.  Team members:
\begin{enumerate}
\item Are cross-functional: includes members with development skills
  (write code or papers/grants), testing skills (e.g.\ data
  collection, test software, proofread papers/grants), and/or domain
  expertise (e.g.\ knowledge or interest in a relevant research area)
\item Are self-organizing and self-managing without externally assigned
  roles
\item Negotiate commitments with the Project Owner, one ``sprint'' at
  a time
\item Have autonomy regarding how to reach commitments
\item Are intensely collaborative
\item Are (ideally) located in one team room (usually this will be the lab)
\item Are (ideally) committed to long-term, consistent lab membership
\item Are (ideally) focused on a single team/project at a time
\item Have a leadership role
\end{enumerate}

\item \textbf{Project Coordinator.} The Project Coordinator facilitates
  the agile research process both directly and indirectly. The Project Coordinator:
\begin{enumerate}
\item Helps to resolve impediments \marginnote{\texttt{NOTE:} The lab
    coordinator role is currently vacant.  The necessary functions of
    the Project Coordinator role will need to be satisfied by other
    project team members during this time.}
\item Creates an environment conducive to team self-organization
\item Captures empirical data to adjust forecasts (e.g.\ weekly Slack
  reports summarizing progress)
\item Shields the team from external interference and distraction to
  keep it ``in the zone''
\item Enforces timelines
\item Has no management authority over the team (anyone with authority
  over the team is by definition not its Project
  Coordinator)
\item Has a leadership role
\item Will usually be the Lab Coordinator %(\href{mailto:contextualdynamics@gmail.com}{\coordinator})
\end{enumerate}

\item \textbf{Collaborator.}  Collaborators are not formally part of
  the project team and generally will not attend regular meetings as
  part of the team.  Collaborators do not have a leadership role in
  the project.  They may carry out one or more of the
  following roles:
\begin{enumerate}
\item Provide data or share equipment \marginnote{\texttt{NOTE:} A
    project may never be held up by a collaborator.  If the
    collaborator fails to provide a promised service, the project team
    must adapt.  If the collaborator fails to meet a non-critical
    deadline, the project will proceed without that component of the
    project.  Involvement as a collaborator is fluid.}
\item Provide occasional consulting services
\item Provide occasional feedback on project results
\item Carry out minor analyses
\item Proofread documents
\item Help with administrative tasks such as scheduling
\item Help with information technology tasks such as computer
  maintenance

\end{enumerate}
By definition, collaborators play a minor role in the project, and they are not
responsible for managing any aspect of the project.  They may become Team
Members if their involvement increases.  Generally, collaborators will
be included in a paper's acknowledgement section, but collaborators
are not normally co-authors.
\end{enumerate}


\newthought{Meetings}

\marginnote{\texttt{TASK:} If you are a senior
    lab member (lab manager, grad student, or postdoctoral
    researcher), discuss with \director~at the beginning of each term
    one thing that you can present at a lab meeting.  Junior lab
    members (undergraduates) are not obligated to present at lab
    meetings, but may schedule a time to do so upon request by
    coordinating with \director.  \textbf{Exception: if you are presenting
    your work outside of the lab, and if it is the first time you are
    presenting that project, you \textit{must} do a practice talk for
    the lab (regardless of whether you are a junior or senior lab member).}}
\noindent  Effective lab communication requires forums for communicating.  As
described below, we use \href{http://www.slack.com}{Slack} to
facilitate non-in-person communications, but Slack cannot replace
in-person meetings.  In fact, our approach is set up to encourage
in-person interactions as often as possible---ideally several times a
week for group projects.  We'll have the following regularly scheduled
meetings:
\begin{enumerate}
\item \textbf{Lab meeting.}  We will have, as a lab, a regular weekly
  1 hour meeting on \textbf{\labmeetingtime}.  The precise
  format of this meeting varies from week to week according to lab
  needs and interests. Attendees: all active lab members.

\item \textbf{Project meetings.}  Several of our collaborative
  projects involve regular coordination with external lab members.
  These are organized on an \textit{ad hoc} basis for each project.
  Attendees: all project team members and any other interested active
  lab members.

  \item \textbf{Hackathons.}  We occasionally organize hackathon
    style events whereby spontaneously organized groups work towards
    one or more very short term projects or goals.  These are
    scheduled on an \textit{ad hoc} basis.  Attendees: all interested
    lab members, any interested member of the Dartmouth community, and
    external collaborators.

\marginnote{\texttt{NOTE:} Each term, you should sign up for a brief beginning-of-term or
    end-of-term meeting time with \director~via \href{https://context-lab.youcanbook.me/}{YouCanBook.Me}.}

  \item \textbf{Beginning-of-term and end-of-term meetings.} At the
    start or end of each term, you should schedule a 15 minute
    meeting slot with \director~to discuss your research
    plans, progress, goals, etc.  It is your responsibility to sign up
    for a slot via \href{https://context-lab.youcanbook.me/}{YouCanBook.Me}.

    \marginnote{\texttt{NOTE:} Department talks and colloquia are
      listed on the \hyperref[sec: scheduling]{PBS Department Events}
      calendar.}
\item \textbf{Department talks and colloquia.} Each week the
  Department of Psychological and Brain Sciences invites internal and
  external researchers to present on a wide variety of research
  topics.  You are encouraged to attend any that seem interesting.
  Attendees: all interested lab and non-lab Dartmouth community
  members.

    \item \textbf{Weekly snippets.} Each week, all
      paid employees must fill out a ``weekly snippet'' Google Form
      with brief answers to the following questions:
      \begin{enumerate}
      \item What did you work on over this past week?
      \item What are you planning to work on this coming week?
      \item What is impeding your progress (if anything)?
      \item Anything else you'd like to add?
      \end{enumerate}
      Weekly snippets are sent out (via Slack) each Monday at 9 AM.
      Whereas weekly snippets are required for all paid employees,
      they are optional for all other lab members.  If you are an
      unpaid employee but are likely to request a letter of
      recommendation, weekly snippets are a good way for me to
      maintain a detailed sense of what you are working on from week
      to week and how you are progressing over time.  (Virtual)
      attendees: all paid lab members and any other lab members who
      want to participate.
\end{enumerate}

%\newpage
\newthought{Getting started in the lab}

\noindent
The very first thing you need to do is to get set up on the following
platforms, which will enable you to interact with the rest of the lab,
download and use the lab's software packages, and accomplish various
necessary administrative tasks: \marginnote{\texttt{TASK:} Create
  (free) Google and GitHub accounts.  Also initiate a request to join our slack workspace via
  \href{https://dartmouth.enterprise.slack.com/workspace/T0W0TEQNA}{this
    link}.}
\begin{enumerate}
\item \href{https://context-lab.slack.com}{\textbf{Slack.}}  This is where
  almost all not-in-person lab communications take place.  It provides
  an interface for asking questions, storing notes, and sharing
  ideas.
\item \href{https://www.github.com}{\textbf{GitHub.}}
  \marginnote{\texttt{TASK:} If you've never used GitHub (Git) before,
    please work through these \href{https://try.github.io/}{GitHub
      Tutorials}. You may also find it useful to refer to this
    \href{https://github.com/ContextLab/lab-manual/blob/master/resources/cheatsheets/git-cheatsheet.pdf}{Git
      cheat sheet} and this
    \href{https://github.com/ContextLab/lab-manual/blob/master/resources/cheatsheets/workflow-of-version-control.pdf}{Git
      workflow} sheet when using Git/GitHub at first.}  This is used
  to manage all code, papers, grants, presentations, and posters.  In
  other words, anything where it'd be useful to track multiple
  versions, anything that we might ultimately want to release to the
  public, and/or anything that multiple lab members will be
  collaborating on.  Each project has one or more GitHub repositories.
\item \href{https://1password.com/}{\textbf{1Password.}}
  \marginnote{\texttt{TASK:} If you are a senior lab member, request a
    1Password invite from \director.} This platform is used by senior
  lab members to manage secure notes and passwords (e.g.\ shared
  software licenses, card numbers and chart strings, etc.).
\item \href{http://www.context-lab.com/}{\textbf{Lab website.}}
  \marginnote{\texttt{TASK:} Submit a photograph (of yourself or some
    other picture or image that you want to represent you) and a 2-3
    sentence biography to \coordinator~so that you can be added to the
    \href{http://www.context-lab.com/people/}{people page} on the lab
    website.  Alternatively, if you do not wish to be included on the
    website, send a note to \coordinator~expressing that you do not
    want to be added to the website.} We use the lab website to
  distribute research materials, describe ongoing work, and provide
  information about our work.
\end{enumerate}

Once you've created those accounts, you can ask any questions through
Slack (use the
\href{https://context-lab.slack.com/messages/general/}{\#general}
channel or the channel specific your project).  Depending on your role
in the lab, you may be added on Slack as a single-channel guest
(access to only one channel) or a full member (access to all lab
channels).  This generally depends on how long you've been in the lab
and/or how many projects you are expecting to interface with.  If you
feel you don't have the appropriate account type, please communicate
your concerns to \director.

\newthought{CITI training}

\noindent Our lab's work seeks to answer questions about human memory.
As such, our research frequently involves interacting with (and
analyzing data collected from) human subjects. It is essential to
understand how to ethically and responsibly maintain the safety,
comfort and privacy of participants when conducting research with
human subjects. Before beginning work on any lab projects, you'll need
to complete an online tutorial through the
\href{https://about.citiprogram.org} {Collaborative Institutional
  Training Initiative (CITI) Program}.

\noindent All lab members are required to complete CITI's
\textbf{Group 2: Social/Behavioral Basic Course} module.  To complete
the module, you will need to:
\begin{enumerate}
  \marginnote{\texttt{NOTE:} If you have previously conducted research
    through a class or another lab, you may already have completed the
    \textbf{Group 2: Social/Behavioral Basic Course} module.  If your
    certificate is valid (i.e., it is less than 3 years old), you may
    send it to \coordinator~without completing the module again.}
\item Create an account on \href{https://about.citiprogram.org}{CITI's
    website}.  If you are a Dartmouth student or employee, choose
  Dartmouth College as your organization affiliation and use your
  \textit{@Dartmouth.edu} email address to create your account.
  Dartmouth will cover your registration fee.
\item Continue the registration process until you reach step 7. Select
  Human Subjects Research. Then select \textbf{Group 2:
    Social/Behavioral Basic Course} as your registration course.  If
  you also need to complete the biomedical module, you will be able to
  add it after completing the social/behavioral module.
\item There are 16 required modules that need to be
  completed. Read/watch the material and complete the quiz at the
  end of each module. You will need to achieve a score of at least an
  80\% to pass. Quizzes may be reviewed and retaken.
\end{enumerate}
\marginnote{\texttt{NOTE:} If you will be working on a project where
  data is collected from DHMC patients, you may need to complete the
  \textbf{Group 1: Biomedical Basic Course} module \textit{in
    addition} to the social/behavioral module.  If you will be working
  on a NSF-funded project, you may also need to complete the
  \textbf{Responsible Conduct of Research (RCR)} module.}

\noindent Once you have finished the training module, send your
\textbf{Completion Certificate} to \coordinator.  To access your
certificate:
\begin{enumerate}
\item Sign into your CITI account and select \textbf{Records} at the
  top of your home page.
\item Select \textbf{View-Print-Share} under \textbf{Completion
    Record}.
\item Under \textbf{Completion Certificate}, select \textbf{View /
    Print}.  You do not need to share your Completion Report.
\item Download the PDF and send it to
  \href{mailto:contextualdynamics@gmail.com}
  {contextualdynamics@gmail.com}.
\end{enumerate}

\newthought{Miscellaneous administrata}

\noindent You can pick up a lab key from Michelle Powers (Moore Hall
administrative office) by
\href{mailto:Michelle.A.Powers@dartmouth.edu}{emailing her} and cc'ing
\director.  You will need to pay a \$5 deposit, which will be returned
to you when you return your key at the end of your tenure in the lab.
If you are the last one in the lab for the day, please be sure to lock
the door when you leave.

\newthought{Starting a new project}

\noindent
Our lab uses a number of project management tools and policies to
promote continuity across projects and lab members.  First, make sure
that your project doesn't already exist (generally this involves
asking \director).

~

The general steps to starting a project are:
\begin{enumerate}

\item Create a Slack channel or decide on existing channel appropriate
  for project use.  \marginnote{\texttt{NOTE:} If you create a new
    Slack channel for your project, invite \director~and other team
    members to join.}
\item Coordinate with \director~to set up a
  \href{https://get.slack.help/hc/en-us/articles/232289568-GitHub-for-Slack}{GitHub
    for Slack} integration between your project and channel by sending
  a link to the GitHub repo in the project's Slack channel with a note
  asking to set up a Slack integration.
\item If your project involves testing human participants, verify with
  \director~that your project is covered under an existing active IRB
  protocol, and that you are listed on the protocol.
  \hyperref[ch:irb]{(A list of active IRB protocols appears at the end
    of this lab manual.)}  Depending on the protocol, you may need to
  complete one or more online training courses to become certified to
  run your experiment.  If no existing protocol is appropriate for
  your study, discuss with \director~whether it would be more
  appropriate to create a new protocol or submit an amendment for an
  existing protocol.
\item Create an initial set of short-term action items for kicking off
  your project and post them to your project's Slack channel.
\end{enumerate}


%\newpage
\newthought{Joining a project}

\noindent To join a project, simply subscribe to the project's Slack
channel and GitHub repository.  All project communications should
either be summarized on Slack or occur through Slack, whenever
possible.  This keeps notes searchable and visible to all team members
(except direct messages, which are useful for non-project-related
private communications between one or more team members).

Note that, as a general rule, you should focus the majority of your
efforts on one project at a time.  This doesn't mean you need to do
the same thing every day (each project has many components and, as
described above, the focus of lab projects will change over time), but
it gives some sense of how you should be allocating your work time.

If you are a part-time employee of the lab, prior experience has shown
that you will most likely be able to make a meaningful contribution to
only a single project at a time.  If you are a full-time employee, you
may choose to devote some of your time to a secondary project (with
the understanding that you will always prioritize your primary
project).  If you want to change which project is your primary
project, or if you want to divide your time across multiple projects,
you should coordinate this with \director~and your team members.


\newthought{Scheduling}
\label{sec: scheduling}

\noindent\marginnote{\texttt{TASK:} Send a Slack direct message to \coordinator~to request
  invites to each of the calendars below for which you require write access---the links in the
  lab manual are read-only!  In your request, please specify which
  calendars you would like write access to.}Our lab's scheduling practices and
policies are intended to facilitate lab member interactions between
ourselves, our collaborators, and our experimental participants.
There are three basic tools the lab uses to organize and schedule
events:
\begin{itemize}
\item \href{http://calendar.google.com}{Google Calendar}:
  \begin{itemize}
  \item We use the
    \href{https://calendar.google.com/calendar/ical/5ta50cfv4uih0a0k8m2di9dhjo\%40group.calendar.google.com/private-ff1338ddce84ac37d5ab682cd94e7f69/basic.ics}{main
      lab calendar} (``Contextual Dynamics Lab'') to keep track of lab-wide events including lab
    meetings, conferences, and activities.  Note: write access is
    granted only to senior lab members-- i.e., lab staff and graduate students.
  \item We use the
    \href{https://calendar.google.com/calendar/ical/dgcv8l8a8s10hfg2s5h0qec0q0\%40group.calendar.google.com/private-4810aed94f818d5748045447ab46c62d/basic.ics}{CDL
      resources calendar} to coordinate the use of shared rooms and
    equipment, such as testing rooms, our EEG systems, eye-tracker,
    hospital equipment, software licenses, etc.  Request write access
    if you may require use of lab equipment.
  \item We use the
    \href{https://calendar.google.com/calendar/ical/h1j06dohcg7v1g2o5tkb7ijhvs\%40group.calendar.google.com/private-239aaf8b4dc60480c90e8d7fc353e229/basic.ics}{out-of-lab
      calendar} to keep track of known absences (e.g. illness, travel,
    holidays, etc.).  \marginnote{\texttt{TASK:} If you are a graduate
      or undergraduate student, please add your schedule to the CDL
      class schedule calendar at the beginning of each term.}  All
    full-time lab members should request write access.
  \item We use the
    \href{https://calendar.google.com/calendar/ical/v2selomg4o6aobua7g922aamqc\%40group.calendar.google.com/private-4377dd542d0d1f6226c9657955b778c5/basic.ics}{CDL
      class schedule calendar} to keep track of course schedules of
    grad students and undergraduates. All student lab members should request
    write access.
  \item We use the \href{https://calendar.google.com/calendar/ical/j6noo2tqahpsoq9na1h16paf3s\%40group.calendar.google.com/private-c3d75bea1ab4605947353d159d3dcd05/basic.ics}{DHMC meetings
  calendar} to keep track of important events and meetings at DHMC.
  \item We use the \href{https://calendar.google.com/calendar/ical/mp1bujb39ud5bgmmnq9r8dbiuk\%40group.calendar.google.com/private-8a3c6919c30e7b118ab3a4fd23aad6cf/basic.ics}{PBS department events calendar} to keep track of talks and events happening in the department.
  \item You may also choose to create project-specific Google
    Calendars, inviting project team members.
  \item When you add an event (in any lab calendar), it is important
    to include the following information as a comment (this does not
    apply to ``out-of-lab'' events):
\begin{itemize}
\item Key contact names and contact information (email or phone)
\item Physical address (where the event will take place)
\item A brief description of the event and/or other relevant
  information
\item Attach any relevant documents via Google Docs
\end{itemize}
\end{itemize}
\item \href{http://www.doodle.com}{Doodle},
  \href{http://www.when2meet.com}{When2Meet}, and
  \href{https://context-lab.youcanbook.me/}{YouCanBook.Me}.  We use
  Doodle and When2Meet to converge on mutually good meeting times that
  fit (as well as possible) with everyone's busy schedules.  Doodle is
  most useful for selecting a date from a large number of options, and
  When2Meet is most useful for selecting a specific time on a
  relatively small number of dates.  YouCanBook.Me is used to sign up
  for meeting slots with \director.  You can book a meeting in a free
  slot through YouCanBook.Me at any time if you would like to meet
  with \director.  Most meetings with \director~happen on Tuesdays.
\end{itemize}

 \newthought{Attendance policy}

 \noindent In general, we expect full time employees to be in the lab
 during ``standard'' working hours---roughly between 9 AM and 5
 PM.\marginnote{\texttt{NOTE:} Our lab is currently operating remotely
   due to the ongoing global pandemic.  During this time, and until
   further notice, you are never expected to be physically \textit{in}
   the lab; you should interpret all references to being ``in'' the
   lab as being ``available'' (e.g., doing lab-related work, signed
   into slack, an so on).  Any in-person lab visits should be
   coordinated with \director~and must comply with Dartmouth's
   \href{https://covid.dartmouth.edu/}{COVID safety protocols}.}  The
 precise range of hours you work is less important to us than putting
 in an effort to help form a cohesive lab culture where lab members
 can interact in person to share ideas, leverage expertise, solve
 problems, etc.  Therefore, even if you end up deciding to shift your
 hours, we'd like you to make a strong effort to be physically present
 in the lab between 1 and 4 PM (prior arrangements notwithstanding;
 e.g. if you have a long commute and we've agreed that you won't come
 in every day, or if you need to occasionally schedule an appointment
 during the 1--4 PM window).  Similarly, if you are a part time
 employee, we'd like you to try to put in your in-the-lab hours during
 the 1--4 PM time window as often as possible.  (This is in addition
 to weekly lab meetings.)

 The lab also abides by Dartmouth's standard paid time off policies
 for benefits-eligible (full-time, non-student) employees.  If you are
 a salaried employee, you can find the official policy
 \href{http://www.dartmouth.edu/~hrs/pdfs/paid_time_off_salaried.pdf}{here},
 and if you are an hourly employee, you can find the official policy
 \href{http://www.dartmouth.edu/~hrs/pdfs/Paid_Time_Off_Hourly.pdf}{here}.
 If you are a student employee, you are generally ineligible for paid
 time off (you can take time off, but you won't normally be paid for
 it).

 If you know that you'll be unable to meet any of these general
 attendance guidelines, please coordinate with \director~to make
 appropriate arrangements.  \textbf{With the above in mind, we abide
   by a ``common sense'' attendance policy that relies on an honor
   system.}\marginnote{\texttt{TASK:} If you are a (paid) hourly
   employee, you'll need to track your hours using the Kronos system.
   \coordinator~can help get you set up with that system.}  If you
 cannot attend a lab event or meeting, your privacy will be respected:
 you do not (generally speaking) need to provide a reason for your
 absence (although you are honor bound not to abuse this system!) but
 you are expected to responsibly manage your schedule so that you get
 your work done and minimize inconvenience to others to the extent
 possible.  The one exception is that if you seem to be abusing this
 system (e.g.\ as determined by your project owner, project
 coordinator, or fellow team members), you may be asked to provide
 additional information (in a way that does not invade your
 privacy---and if you are worried that this policy is overly
 intrusive, please bring your concerns to \director).  Here's the
 official lab attendance policy:
\begin{itemize}
\item It is your responsibility to provide notice, well in advance, to
  anyone your absence will affect (e.g. \director, people you're
  scheduled to meet with, etc.).  The best way to do this is via email
  or Slack, along with adding your absence to the
  \hyperref[sec: scheduling]{out-of-lab calendar}.
\begin{itemize}
\item You are responsible for accounting for your planned absences
  when we discuss project schedules and goals.  If you agree to take
  on work or to meet a deadline, you're responsible for it until you
  make alternative plans with your team!
\item Prolonged (more than 1 day, excluding weekends and
  lab-related absences) planned
  absences should be scheduled at least 1 week in advance, and ideally
  2 weeks in advance.
\item Brief (one day) absences (excluding weekends, and lab-related
  absences) should be scheduled as far in advance as possible, but at
  least at the beginning of the week, emergencies notwithstanding.
\end{itemize}

\item If you are feeling sick, \textit{do not come into the lab}.  We
  can arrange virtual meetings (if you are feeling well enough) or
  re-schedule as needed.  Your recovery, and the health and safety of
  the lab, are the top priorities.

\item If you need to be out of the lab for an unexpected
  non-illness-related emergency, simply give as much notice and
  information as possible.

\item You are expected to attend all lab meetings and other regularly
  scheduled meetings that are directly relevant to your work in the
  lab.

\item If you are scheduled to present at a conference (i.e.\ you
  submit an abstract and it is accepted as a talk or poster), you are
  expected to attend the conference to present your work.  In the
  extremely rare event that an emergency situation arises that would
  prevent you from presenting as scheduled, you are expected to make
  alternative arrangements (e.g.\ by arranging for a co-author to
  present in your place).

\item You are strongly encouraged (but never required) to attend
  relevant journal clubs, PBS talks, DHMC meetings and talks, thesis
  defenses, and other relevant lab and/or Dartmouth-affiliated events.
  If you are overwhelmed with other work, have a conflicting meeting,
  are running an experimental participant, or are out of the lab for
  other reasons, you do not need to provide a reason for your absence
  (unless you're presenting or are otherwise playing a key role).
\end{itemize}

\newthought{Compensation and benefits}

If you are a non-student full-time on-campus employee, it's likely
that you're eligible for Dartmouth benefits, such as medical
insurance, dental insurance, life insurance, etc.  You can read more
about the comprehensive benefits package
\href{http://www.dartmouth.edu/~hrs/benefits/}{here}.

Dartmouth also sponsors various health-benefits programs (for all
members of the Dartmouth community).  For example, you are likely
eligible to get a free (or subsidized) fitness tracker, fitness
equipment, race fees, gym membership, etc.  You can also earn cash (up
to \$400/year) for meeting your fitness goals.  Go1
\href{http://join.virginpulse.com/dartmouth/}{here} to learn more or
sign up for this program.

If you are a student employee, you may be paid or unpaid.  In general,
full-time student employees are paid and part-time student employees
are unpaid until they have been a full-time employee for at least one
term.  Your precise level of compensation will depend on your
position, how your work in the lab is funded, your prior research
experience, etc.  If you have comments, questions, or concerns about
your compensation, please discuss them with \director.

\newthought{Interpersonal issues}

If you are experiencing an interpersonal issue with another lab member
or community member and are having trouble resolving it on your own
(or feel unsafe resolving it on your own), please seek out assistance
from \director, your Project Owner, your Project Coordinator, or one
of the Dartmouth community resources described below as early as
possible.

All lab members, regardless of position or status, are protected by
(and must abide by) Dartmouth's human resources policies.  This means
behaving professionally and respectfully towards others (including,
but not limited to, your fellow lab members).  On (hopefully) rare
occasions, despite your best efforts, you may find yourself in an
interpersonal situation that you feel unable to resolve on your own.
You have many resources at your disposal to help get you back on
track.

The \href{http://www.dartmouth.edu/~hrs/}{Office of Human Resources}
provides assistance and resources to all faculty, staff, retirees, and
prospective employees. The
\href{http://www.dartmouth.edu/~seo/}{Student Employment Office}
provides a similar suite of services to student employees.  The
\href{http://www.dartmouth.edu/~ombuds/}{Dartmouth Ombuds Office} also
provides confidential and informal assistance in resolving concerns
related to interpersonal issues.  Please also see the section on
\hyperref[sec:interpersonal]{resources for resolving interpersonal
  issues}.


 \newthought{Lab resources}

 \noindent As with most academic research labs, we (sadly!) must
 conduct our research within a limited research budget.  In practice,
 the important thing is to communicate with \director~before you spend
 (or commit to spending) lab funds.

 Generally, the lab's financial policy is the following: we will do
 whatever is possible to ensure you have the equipment and resources
 you \textit{need} to do your best work.  If you can adequately
 justify an expense and sufficient funds are available, then we will
 spend what it takes to get the job done.  If you cannot justify an
 expense, or if the lab does not have sufficient funds, then we will
 need to get creative by figuring out how to get the job done anyway
 on a seemingly too-small budget.  Usually we'll find ourselves
 somewhere in the middle of this continuum, which will help us to
 stretch our limited budget as far as possible while not making
 ourselves crazy or losing too much productivity in the process.

 Some of our projects are intended to be self-funded and/or to support
 other projects (e.g.\ StockProphet).  Any use of project-generated
 funds should be discussed with \director.

 \subsection{Computers}
 All lab members need a computer to get their work done.  We generally
 prefer to use Macs, as this maximizes compatibility across lab
 members.  Depending on your expected role in the lab and the
 specifics of your project, the lab may provide a computer to you, or
 you may be expected to use your personal computer to complete your
 work.  Any equipment purchased by the lab, including personal
 computers, is the official property of the Contextual Dynamics Lab
 and should be treated as such.  All equipment must be returned to the
 lab when your association with the lab is complete.

 In addition to personal computers, we also maintain a lab account on
 Dartmouth's \href{http://techdoc.dartmouth.edu/discovery/}{Discovery
   Supercomputing Cluster}.  In addition to having access to the
 compute nodes shared amongst the entire Dartmouth community, we have
 purchased several dedicated servers and a powerful head node that is
 shared with the \href{http://www.cosanlab.com/}{COSAN Lab} and the
 \href{http://www.dartmouth.edu/~bil/}{Dartmouth Brain Imaging Lab}.
 When you link your Discovery account with the lab, you will
 automatically have access to those additional computing resources.
 We use Discovery for our most computationally intensive work.
 \href{https://rc.dartmouth.edu/index.php/discovery-overview/accessing-the-cluster}{This
   link} contains instructions for creating an account and
 accessing % potentially add link to cluster tutorial when completed
 the Discovery cluster.  Our suggested workflow is to do non-intensive
 computations and analyses on your personal desktop or laptop
 computer, and to offload more intensive analyses to Discovery.  The
 lab's code repository includes
 \href{https://github.com/ContextLab/cluster-tools-dartmouth}{sample
   Python scripts} for running analyses on Discovery.

 \subsection{Other research equipment}
 Many research projects require specialized research equipment (e.g.\
 for neuroimaging using fMRI, EEG, ECoG, etc.).  Some of the necessary
 research equipment is owned by the Contextual Dynamics Lab, and other
 equipment is shared with other labs affiliated with PBS or DHMC.  All
 equipment should be treated with care and respect.  Any malfunctions
 should be reported immediately.

\subsection{Repository of shared lab papers}
Our lab maintains a Dropbox repository of PDFs for internal use by lab
members and affiliates. Contact \director~for a link (not to be shared
publicly).

 \subsection{Travel policy}
 \marginnote{\texttt{NOTE:} To obtain funding for a scientific
   conference, \textit{prior to submitting your abstract}, you must
   (a) briefly describe or discuss how attending the conference fits
   in with your goals and/or plans, (b) provide confirmation that you
   have applied for some form of external funding, and (c) provide an
   approximate travel budget, including all registration fees,
   tickets, meals, etc.  Your budget should not include lab events
   (e.g.\ lab dinners), which are covered by a separate mechanism.
   Your budget must be approved by \director~prior to submitting an
   abstract in order for the lab to cover your expenses.  (Your
   expenses will be covered up the agreed-upon amount, at which point
   you are responsible for making your travel plans accordingly,
   submitting receipts, etc.)  To initiate a request for travel
   funding, join the \texttt{\#receipts} channel on Slack and use the
   workflow (lightning bolt in the lower left) menu to start a
   ``Conference \$ request''.}A major component of doing scientific
 research is communicating with other scientists.  The Contextual
 Dynamics Lab regularly presents at several international scientific
 conferences.  If you are presenting your work from the lab (i.e., you
 are the presenting author for a talk or poster), then your travel
 expenses and conference registration fees will be guaranteed by the
 lab, \textit{under the assumption that you will also make reasonable
   efforts to seek out alternative sources of travel funding} (e.g.\
 through PBS, other internal Dartmouth sources, applying for travel
 awards, using personal grants like NRSAs or NSF fellowships, etc.).
 You are also expected to keep costs low (e.g.\ fly economy class,
 seek out cheaper tickets, stay in reasonably priced hotels, share a
 room with other lab members, etc.).  By the same token, we also want
 to be cognizant of your comfort and time, and it is not always
 necessary to use the cheapest option.  More specific travel
 guidelines will be given on a per-conference or per-trip basis.

 If you are not presenting your work (or if you're presenting non-lab
 work), but you are a senior lab member, then the lab may cover your
 travel expenses to a limited number of conferences each year.  These
 should be discussed on a case-by-case basis with \director.

 If you are a junior lab member not presenting your work, the lab will
 generally not pay for you to attend conferences.  However, if you are
 interested in attending a conference, and you aren't able to secure
 funding through non-lab sources, you should discuss your options with
 \director.

 \subsection{Making a poster}
 \marginnote{\texttt{NOTE:} If you commit to presenting a poster at a
   scientific conference, you agree to prepare a complete draft of
   your poster \textit{at least two weeks in advance} of the
   conference, and to complete a final draft of your poster (that
   incorporates feedback from \director) \textit{at least one week in
     advance} of the conference.  If you do not meet these deadlines,
   you may be required to withdraw your submission and/or cover any
   conference-related expenses that you incurred, at \director's
   discretion.} The preferred methods for creating posters are to use
 \href{https://github.com/deselaers/latex-beamerposter}{LaTeX
   BeamerPoster}, Adobe Illustrator, or
 \href{https://inkscape.org/en/}{Inkscape}.  The lab has several
 example poster templates on file, but these are not organized well
 yet.  Your best bet is to ask \director~for an old poster and modify
 it.  You can also create a file from scratch.  Ensure that any images
 are either vector graphics, or bitmaps (.png, .jpeg, .tiff, etc.) at
 sufficiently high resolutions (at least 300 dpi).


 \subsection{Poster printing}
 \marginnote{\texttt{NOTE:} Be sure to coordinate with
   \director~regarding the funding source covering your poster
   printing cost prior to going to print your poster.}  There are two
 on-campus poster printers.  One is in the Map Room of Baker Library.
 More information may be found
 \href{http://www.dartmouth.edu/~library/maproom/printingfaq.html}{here}.
 The Map Room printer should be used in most cases.  It is important
 to schedule your printing time as far in advance as possible,
 particularly before conferences when many people will want to print.
 Advanced planning can help us avoid the additional costs associated
 with off-campus printers.  The Department of Psychological and Brain
 Sciences covers conference poster printing costs for PBS labs and
 students (coordinate with \coordinator~to obtain the PBS Chart String
 for poster printing).


 \subsection{Publication costs}
 All costs related to lab publications will be fully covered by the
 lab.  \coordinator~can help facilitate these payments.

 \subsection{Subject compensation}
 Most in-lab experimental subjects will be compensated via t-points.
 Coordinate with \coordinator~in order to receive a t-point allocation
 well in advance of the start of the term you wish to run
 participants.

 \marginnote{\texttt{NOTE:}In some cases, external funding sources may
   be available to cover cash subject payments
   (e.g. \href{https://students.dartmouth.edu/ugar/research/programs/honors-thesis-grant}{undergraduate
     Honors Thesis Grants} or
   \href{http://neukom.dartmouth.edu/programs/neukom-scholars.html}{Neukom
     Scholars grants}). Coordinate with \director~to determine whether
   alternative funding sources are available for your project.}  Cash
 subject payments for lab research projects will in most cases be
 fully covered by the lab (see the \texttt{NOTE}).  Subject payment
 guidelines are generally found in the IRB approval documentation
 relevant to your project.  For Mechanical Turk experiments, a subject
 payments budget should be approved prior to beginning the experiment.
 Cash payments may be made via petty cash, which is managed by
 \coordinator.


\newthought{Who to go to with questions}

\noindent This section contains guidelines of where to direct
questions related to technical and interpersonal issues that may
arise, or any issues you encounter not already listed in the sections
below. In general, resolving \textit{technical} issues within the lab
saves time, but there will inevitably be some we cannot resolve on our
own. In general, interpersonal issues should be resolved in accordance
with your comfort, but it is important to be aware of the below
resources should you encounter a situation in which you want to use
them.

\subsection{Tech issues}
\begin{itemize}
\item For issues related to our lab's fileserver, seek assistance from
  \director.

\item For issues related to Slack, send a direct message to @Ryan Sokol.

\item For issues with other lab-owned equipment (e.g.~computers,
  research equipment, etc.) \coordinator~or the senior lab member
  actively using the equipment should be your first-sought resource.

\item For issues related to the Discovery computing cluster,
  contact \href{mailto:John.P.Hudson@Dartmouth.edu}{John Hudson}.

\item For other general Dartmouth-specific IT questions or issues,
  \href{help@dartmouth.edu}{open a ticket} with the Information,
  Technology, and Consulting office, or contact
  \href{mailto:Andrew.C.Knutsen@Dartmouth.edu}{Andrew Knutsen}.
\end{itemize}

\subsection{Lab documents}
\begin{itemize}
\item For questions about non-paper documents (e.g. receipts, consent
  forms, IRB protocols, the lab website, etc.), talk to \coordinator.

\item For questions on papers or posters, talk to \director.
\end{itemize}

\subsection{Lab policy}
\begin{itemize}
\item If you have a question about the lab's policies, first try to
  find the answer in the
  \href{https://github.com/ContextLab/lab-manual/tree/master/lab_manual.pdf}{Lab
    Manual}.

\item If you feel your question is not adequately answered in the Lab
  Manual, post your question on Slack (or, if it's a private issue,
  talk to \director).

\item If you think your question is likely to be of general interest,
  update the lab manual to reflect the answer.
\end{itemize}

\subsection{Specific methods}
\begin{itemize}
\item If you have questions about specific methods related to a
  project, ask a senior lab member working on the project.

\item It may also be helpful to check the
  \href{https://github.com/contextLab/cdl-tutorials}{CDL tutorials}
  repo, and contact the author(s) or contributors of a related
  tutorial.

\item You can also post your question in the
  \href{https://context-lab.slack.com/messages/computrons/}{\#computrons}
  channel on Slack.

\item Remember, \href{https://www.google.com/}{Google} and
  \href{https://stackoverflow.com/}{Stack Overflow} are your friends!
  Often other people have encountered (and solved!) similar problems,
  and sometimes they even share the answers online!

\item If you still cannot find an answer to your question, reach out to \director.
\end{itemize}

\subsection{Interpersonal issues}\label{sec:interpersonal}
\begin{itemize}
\item Clear, direct communication is often the best way to address
  interpersonal issues. However, if you are having trouble resolving
  something (or feel uncomfortable doing so on your own) you should
  reach out to one of the following resources:

\item Senior lab members (e.g. your immediate supervisor)

\item \director

\item The \href{https://pbs.dartmouth.edu/people}{PBS department chair}

\item The
  \href{https://students.dartmouth.edu/undergraduate-deans/}{Undergraduate
    Deans Office} or the
  \href{mailto:Graduate.and.Advanced.Studies@Dartmouth.edu}{Graduate
    and Professional Schools Deans Office}

\item Dartmouth's \href{https://www.dartmouth.edu/~hrs/}{Office of
    Human Resources}

\item Dartmouth's
  \href{https://www.dartmouth.edu/~eap/}{Faculty/Employee Assistance
    Program}

\item Dartmouth's \href{https://sexual-respect.dartmouth.edu/}{Title
    IX office} (for concerns about sexual misconduct, harassment, or
  assault)

\item
  \href{https://www.hanovernh.org/hanover-police-department}{Hanover
    Police} (if criminal activity is involved)

\item WISE, a 24-hour crisis hotline for free and confidential
  services to address domestic and sexual violence and stalking
  (\href{tel:18884970516}{866-348-9473})

\end{itemize}

\subsection{Ethics issues and reporting}
\begin{itemize}

\item Dartmouth's
  \href{https://www.dartmouth.ethicspoint.com}{Compliance and Ethics
    Hotline} (\href{tel:18884970516}{888-497-0516})
\item Safety and Security's
  \href{https://www.dartmouth.edu/~security/services/forms/anonreport.html}{anonymous
    reporting form}

\end{itemize}


\chapter{Internal Review Board (IRB) approvals}\label{ch:irb}
Experimental \marginnote{\texttt{NOTE:} prior to running any
  non-lab-member participants in your study, you must coordinate with
  \coordinator~to verify that (a) your experiment has been approved by
  the IRB and (b) your name is specifically listed on the associated
  protocol.} protocols and IRB approval forms are maintained and
coordinated through \href{https://rapport.dartmouth.edu/}{RAPPORT}.
Typically this system is accessed directly by \director, but if you
feel you need access to RAPPORT then let \director~know.

 \newthought{List of active protocols}

 \begin{enumerate}
 \item \textbf{Efficient Learning (STUDY00029685).}  Used for all
   behavioral learning and memory experiments for participants run in
   the lab or via Amazon Mechanical Turk.  The protocol also covers
   fitness-related studies (e.g.\ incorporating fitness tracker data
   and other on-body and remote biosensors including scalp EEG and
   eye-trackers, and/or performing physical tasks like riding on a
   stationary bicycle or running in place).
    \item \textbf{fMRI Efficient Learning (STUDY00030020).}  Analog of
      the Efficient Learning protocol, but allows for fMRI data
      collection.
    \item \textbf{Electrophysiological Localization of Human Brain
        Function (STUDY00012495).}  Hospital protocol used for
      studying electrocorticographic activity in epilepsy patients.
    \item \textbf{Network dynamics in human epilepsy (STUDY00029400).}
      Hospital protocol used for studying network dynamics inferred
      via electrocorticographic activity in epilepsy patients.
    \item \textbf{Perceptual and Memory Dynamics (STUDY00031514).}
      Used for studies investigating the relationship between
      perceptual and memory metrics and symptoms associated with
      common psychiatric disorders.

   \end{enumerate}

 \newthought{List of inactive protocols}
 \begin{enumerate}
 \item \textbf{EEG Efficient Learning (STUDY00029881).} Analog of the
   Efficient Learning protocol, but allows for scalp EEG data
   collection.  This protocol is redundant with the revised
   Efficiently Learning protocol, which now allows for scalp EEG data
   collection as well.
   \end{enumerate}

   \newthought{Testing procedure}

   When you run an experimental participant, you must have them sign a
   consent form.  After each day of testing (or more frequently), you
   should give all signed consent forms to \coordinator.  Consent
   forms are kept in a locked file cabinet in \director 's office.

   All experimental data must be stored securely, as per IRB
   guidelines.  All lab computers employ disk encryption and password
   protection.  If you need to copy unpublished data onto a non-lab
   computer, you need to verify (by checking in with \director) that
   your computer satisfies our data security requirements.  \textit{No
     personally identifiable data about our participants may ever be
     shared outside of the lab.}

\chapter{Lab members and alumni}\label{ch:members}
\begin{fullwidth}
\subsection{Current lab members}\label{sec:curr_members}
\newthought{PI}
\bigskip

\enskip Jeremy R. Manning (2015 -- )


\newthought{Graduate Students}
\begin{multicols}{2}\raggedcolumns
\begin{list}{\quad}{}
\item Kirsten Ziman (2017 -- )
\item Paxton Fitzpatrick (2021 -- )
\item Xinming Xu (2021 -- )
\end{list}
\end{multicols}


\newthought{Research Assistants}
\begin{multicols}{2}\raggedcolumns
\begin{list}{\quad}{}
\item Mark Taylor (2021 -- )
\end{list}
\end{multicols}


\newthought{Undergraduate RAs}
\begin{multicols}{2}\raggedcolumns
\begin{list}{\quad}{}
\item Tudor Muntianu (2019 -- )
\item Chelsea Uddenberg (2020 -- )
\item Greg Han (2020 -- )
\item Esme Chen (2020 -- )
\item Chris Long (2020 -- )
\item Ethan Adner (2020 -- )
\item Chris Jun (2020 -- )
\item Chris Suh (2020 -- )
\item Tyler Chen (2020 -- )
\item Darren Gu (2020 -- )
\item Aidan Adams (2021 -- )
\item Damini Kohli (2021 -- )
\item Natalie Schreder (2021 -- )
\item Swestha Jain (2021 -- )
\item Will McCall (2021 -- )
\item Daniel Carstensen (2021 -- )
\item Ansh Patel (2021 -- )
\item Anna Mikhailova (2021 --)
\item Thomas Corrado (2021 --)
\item Brian Chiang (2021 -- )
\item Kevin Cao (2022 -- )
\item Goutham Veeramachaneni (2022 -- )
\item Zachary Somma (2022 -- )
\item Dawson Haddox (2022 -- )
\item Jessna Brar (2022 -- )
\end{list}
\end{multicols}

\subsection{Lab alumni}
\newthought{Postdoctoral Researchers}
\begin{multicols}{2}\raggedcolumns
\begin{list}{\quad}{}
\item Andrew Heusser (2016 -- 2018)
\item Gina Notaro (2017 -- 2018)
\end{list}
\end{multicols}

\newthought{Graduate Students}
\begin{multicols}{2}\raggedcolumns
  \begin{list}{\quad}{}
  \item Lucy Owen (2016 -- 2021)
  \item Caroline Lee (2019 -- 2021)
  \end{list}
  \end{multicols}

\newthought{Lab Managers}
\begin{multicols}{2}\raggedcolumns
\begin{list}{\quad}{}
\item Kirsten Ziman (2016 -- 2017)
\item Emily Whitaker (2017 -- 2018)
\item Paxton Fitzpatrick (2018 -- 2021)

\end{list}
\end{multicols}


\newthought{Research Assistants}
\begin{multicols}{2}\raggedcolumns
\begin{list}{\quad}{}
\item Max Bluestone (2018 -- 2019)
\item Maddy Lee (2020 -- 2021)
\item Xinming Xu (2019 -- 2021)
\end{list}
\end{multicols}


\newthought{Undergraduate RAs}
\begin{multicols}{2}\raggedcolumns
\begin{list}{\quad}{}
\item Jessica Tin (2016)
\item Gal Perlman (2016)
\item Peter Tran (2016)
\item Sheherzad Mohydin (2016)
\item Joseph Finkelstein (2016)
\item Aman Agarwal (2016)
\item Clara Silvanic (2016)
\item Jake Rost (2016)
\item Aamuktha Porika (2016 -- 2017)
\item Allison Frantz (2016 -- 2017)
\item Marisol Tracy (2016 -- 2017)
\item Wei Liang Samuel Ching (2016 -- 2017)
\item Armando Ortiz (2017)
\item Christina Lu (2017)
\item Campbell Field (2016 -- 2018)
\item Bryan Bollinger (2017 -- 2018)
\item Stephen Satterthwaite (2017 -- 2018)
\item Ann Carpenter (2018)
\item Kirsten Soh (2018)
\item Rachael Chacko (2018)
\item Darya Romanova (2018)
\item Iain Sheerin (2018)
\item Mustafa Nasir-Moin (2018)
\item Seung Ju Lee (2018)
\item Paxton Fitzpatrick (2017 -- 2019)
\item Maddy Lee (2016 -- 2020)
\item Alex Martinez (2018 -- 2020)
\item William Chen (2019 -- 2020)
\item Aaron Lee (2019 -- 2020)
\item Shane Hewitt (2019 -- 2020)
\item Sarah Park (2019 -- 2020)
\item Austin Zhang (2020)
\item Vivian Tran (2020)
\item Luca Lit (2020)
\item Kelly Rutherford (2020)
\item Helen Lu (2020)
\item Kunal Jha (2020)
\item Will Baxley (2018 -- 2021)
\item Daniel Ha (2021)
\item Shane Park (2019 -- 2021)
\item Tehut Biru (2020 -- 2021)

\end{list}
\end{multicols}
\end{fullwidth}


\chapter{Checklist and signature page}

   By signing below, I certify that I have completed the following tasks:
   \begin{todolist}
     \item I have joined the lab's Slack account and worked through the
    orientation workflow to share my accounts and information with
    \director.  (Click the lightning bolt button in the lower left of
    the \#general channel and select `Join the lab!' to start the
    onboarding process.)
    \item I agree to check Slack
    regularly (at \textit{least} once per normal business day, excluding days I
    mark on the out-of-lab calendar) and to respond to messages and
    requests in a timely manner (within a few hours during normal
    business hours, or early the next business day if messages are
    received after hours).
  \item I have created a GitHub account and have been added to the appropriate lab
    GitHub group(s).  If I am unfamiliar with Git, I have gone through
    the GitHub tutorials.
  \item I am proficient in \LaTeX, allowing me to edit and understand
  this lab manual's source code.
\item I have submitted a pull request to the
  \href{https://github.com/ContextLab/lab-manual}{GitHub repository}
  that adds my name to the \hyperref[sec:curr_members]{lab members}
  section, along with any other updates I think would be helpful.  I
  have also recompiled the PDF of the lab manual by running
  \texttt{pdflatex} \textbf{twice}, and have ensured that (a) the PDF compiled
  without errors, and (b) the table of contents was generated correctly.
  \item I have access to the following lab calendars:
    Contextual Dynamics Lab, Out of lab, CDL Resources, CDL class
    schedule (if applicable), DHMC Meetings (if
    applicable), and PBS department events.
  \item I have been added to the lab's 1password account if I am a
    senior lab member, or I have not been added because I am a junior
    lab member.
  \item I have completed the required CITI program module(s) and have
    sent my Completion Certificate(s) to \coordinator.
  \item I am an unpaid or salaried employee, or I am an hourly
    employee and have gotten myself set up on Kronos to track my paid
    hours.
    \item If I am a paid employee (including graduate students), or if
      I would like to be paid, I agree to regularly seek out and
      coordinate with \director~about relevant funding opportunities
      that could help support me and/or my research.
  \item I have submitted a photograph (of myself, or a representative
    scene or object) along with a brief (2-3 sentence), professionally
    worded biography, to be included on the lab website.
    Alternatively, I indicated (via Slack message to \coordinator) my
    preference to not be included on the lab website.
  \item I agree to regularly attend lab meetings (unless I have a
    course conflict and have informed \director).
  \item I agree to set up a beginning-of-term or end-of-term meeting
    with \director~at least once per term.
    \item I agree to send \director~regular updates regarding my
      progress and work, including weekly snippets, meetings, Slack
      discussions, and/or GitHub updates.
      \item If/when I decide to end my affiliation with the lab, I
        will notify \director~and any other project members that may
        be affected by my departure.  I will also coordinate with
        \director~and other relevant lab members to facilitate a smooth
        transfer of your project-related roles and responsibilities to
        other lab members.  This includes a discussion about potential
        authorship on future publications related to the work.
  \item I have read
    \href{https://www.sciencemag.org/careers/2018/11/what-can-we-learn-dartmouth}{Leah
      Somerville's commentary} on speaking out in toxic or abusive
    environments.
  \item I have reviewed the
    \href{https://oir.nih.gov/sites/default/files/uploads/sourcebook/documents/ethical_conduct/guidelines-authorship_contributions.pdf}{NIH
      guidelines for authorship}.
  \item I agree to abide by the lab's bill of rights and
    responsibilities, and to follow official lab practices and
    policies.
  \item I have carefully read through the entire lab manual, and I
    have checked off all of the above items to indicate that I have
    carried out the indicated tasks.
  \item I will send a digital copy (PDF) of the final two pages of
    this manual, including checked-off list items, my digital
    signature, and today's date, to
    \href{mailto:contextualdynamics@gmail.com}{contextualdynamics@gmail.com}.
\end{todolist}

\vspace{0.25in}
\begin{tabular}{@{}p{3in}p{1in}@{}}
\hrulefill & \hrulefill \\
Signature & Date\\
\end{tabular}


% Citation example \cite{Tufte2001}, notice how the citation is in the margin. This is an example of how to add something to the index at the end of the document.\index{citation}

% \newthought{Example of} the \texttt{newthought} command for starting new sections. Typography examples: \allcaps{all caps} and \smallcaps{small caps}.

%------------------------------------------------

% \section{Figures}

% \lipsum[1]

% \begin{marginfigure}
% \includegraphics[width=\linewidth]{helix}
% \caption{This is a margin figure. The helix is defined by $x = \cos(2\pi z)$, $y = \sin(2\pi z)$, and $z = [0, 2.7]$. The figure was drawn using Asymptote (\url{http://asymptote.sf.net/}).}
% \label{fig:marginfig}
% \end{marginfigure}

% \lipsum[2]

% \begin{figure*}[h]
% \includegraphics[width=\linewidth]{sine.pdf}
% \caption{This graph shows $y = \sin x$ from about $x = [-10, 10]$.
% \emph{Notice that this figure takes up the full page width.}}
% \label{fig:fullfig}
% \end{figure*}

% \lipsum[3]

% %------------------------------------------------

% \section{Tables} \marginnote{This is a random margin note. Notice that there isn't a number preceding the note, and there is no number in the main text where this note was written. Use \texttt{sidenote} to use a number.}

% \lipsum[4]

% \begin{table} % Add the following just after the closing bracket on this line to specify a position for the table on the page: [h], [t], [b] or [p] - these mean: here, top, bottom and on a separate page, respectively
% \centering % Centers the table on the page, comment out to left-justify
% \begin{tabular}{l c c c c c} % The final bracket specifies the number of columns in the table along with left and right borders which are specified using vertical bars (|); each column can be left, right or center-justified using l, r or c. To specify a precise width, use p{width}, e.g. p{5cm}
% \toprule % Top horizontal line
% & \multicolumn{5}{c}{Growth Media} \\ % Amalgamating several columns into one cell is done using the \multicolumn command as seen on this line
% \cmidrule(l){2-6} % Horizontal line spanning less than the full width of the table - you can add (r) or (l) just before the opening curly bracket to shorten the rule on the left or right side
% Strain & 1 & 2 & 3 & 4 & 5\\ % Column names row
% \midrule % In-table horizontal line
% GDS1002 & 0.962 & 0.821 & 0.356 & 0.682 & 0.801\\ % Content row 1
% NWN652 & 0.981 & 0.891 & 0.527 & 0.574 & 0.984\\ % Content row 2
% PPD234 & 0.915 & 0.936 & 0.491 & 0.276 & 0.965\\ % Content row 3
% JSB126 & 0.828 & 0.827 & 0.528 & 0.518 & 0.926\\ % Content row 4
% JSB724 & 0.916 & 0.933 & 0.482 & 0.644 & 0.937\\ % Content row 5
% \midrule % In-table horizontal line
% \midrule % In-table horizontal line
% Average Rate & 0.920 & 0.882 & 0.477 & 0.539 & 0.923\\ % Summary/total row
% \bottomrule % Bottom horizontal line
% \end{tabular}
% \caption{Table caption text} % Table caption, can be commented out if no caption is required
% \label{tab:template} % A label for referencing this table elsewhere, references are used in text as \ref{label}
% \end{table}

%----------------------------------------------------------------------------------------

\mainmatter

%----------------------------------------------------------------------------------------
%	CHAPTER 1
%----------------------------------------------------------------------------------------

%\chapter{Chapter 1 Title}
%\label{ch:1}

%------------------------------------------------

% \section{Section 1 - Fullwidth Environment Example}

% \begin{fullwidth}
% \lipsum[5]
% \end{fullwidth}

% \subsection{Subsection 1}

% \lipsum[6-7]

% \subsection{Subsection 2}

% \lipsum[7-8]



%----------------------------------------------------------------------------------------

\backmatter

%----------------------------------------------------------------------------------------
%	BIBLIOGRAPHY
%----------------------------------------------------------------------------------------

\bibliography{bibliography} % Use the bibliography.bib file for the bibliography
\bibliographystyle{plainnat} % Use the plainnat style of referencing

%----------------------------------------------------------------------------------------

\printindex % Print the index at the very end of the document


\end{document}
