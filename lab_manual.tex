%%%%%%%%%%%%%%%%%%%%%%%%%%%%%%%%%%%%%%%%%
% Tufte-Style Book (Minimal Template)
% LaTeX Template
% Version 1.0 (5/1/13)
%
% This template has been downloaded from:
% http://www.LaTeXTemplates.com
%
% License:
% CC BY-NC-SA 3.0 (http://creativecommons.org/licenses/by-nc-sa/3.0/)
%
% IMPORTANT NOTE:
% In addition to running BibTeX to compile the reference list from the .bib
% file, you will need to run MakeIndex to compile the index at the end of the
% document.
%
%%%%%%%%%%%%%%%%%%%%%%%%%%%%%%%%%%%%%%%%%

%----------------------------------------------------------------------------------------
%	PACKAGES AND OTHER DOCUMENT CONFIGURATIONS
%----------------------------------------------------------------------------------------

\documentclass{tufte-book} % Use the tufte-book class which in turn uses the tufte-common class

\hypersetup{colorlinks} % Comment this line if you don't wish to have colored links

\usepackage{microtype} % Improves character and word spacing

%\usepackage{lipsum} % Inserts dummy text

\usepackage{booktabs} % Better horizontal rules in tables

\usepackage{graphicx} % Needed to insert images into the document
\graphicspath{{graphics/}} % Sets the default location of pictures
\setkeys{Gin}{width=\linewidth,totalheight=\textheight,keepaspectratio} % Improves figure scaling

\usepackage{fancyvrb} % Allows customization of verbatim environments
\fvset{fontsize=\normalsize} % The font size of all verbatim text can be changed here

\newcommand{\hangp}[1]{\makebox[0pt][r]{(}#1\makebox[0pt][l]{)}} % New command to create parentheses around text in tables which take up no horizontal space - this improves column spacing
\newcommand{\hangstar}{\makebox[0pt][l]{*}} % New command to create asterisks in tables which take up no horizontal space - this improves column spacing

\usepackage{xspace} % Used for printing a trailing space better than using a tilde (~) using the \xspace command

\newcommand{\monthyear}{\ifcase\month\or January\or February\or March\or April\or May\or June\or July\or August\or September\or October\or November\or December\fi\space\number\year} % A command to print the current month and year

\newcommand{\openepigraph}[2]{ % This block sets up a command for printing an epigraph with 2 arguments - the quote and the author
\begin{fullwidth}
\sffamily\large
\begin{doublespace}
\noindent\allcaps{#1}\\ % The quote
\noindent\allcaps{#2} % The author
\end{doublespace}
\end{fullwidth}
}

\newcommand{\ourschool}{OUR SCHOOL}

\newcommand{\blankpage}{\newpage\hbox{}\thispagestyle{empty}\newpage} % Command to insert a blank page

\usepackage{makeidx} % Used to generate the index
\makeindex % Generate the index which is printed at the end of the document

%----------------------------------------------------------------------------------------
%	BOOK META-INFORMATION
%----------------------------------------------------------------------------------------

\title{Laboratory Manual} % Title of the book

\author{Jeremy R. Manning, Ph.D.} % Author

\publisher{Computational Memory Laboratory, \ourschool} % Publisher

%----------------------------------------------------------------------------------------

\begin{document}

\frontmatter

%----------------------------------------------------------------------------------------
%	EPIGRAPH
%----------------------------------------------------------------------------------------

\thispagestyle{empty}
%\openepigraph{Quotation 1}{Author, {\itshape Source}}
%\vfill
%\openepigraph{Quotation 2}{Author}
%\vfill
%\openepigraph{Quotation 3}{Author}

%----------------------------------------------------------------------------------------

\maketitle % Print the title page

%----------------------------------------------------------------------------------------
%	COPYRIGHT PAGE
%----------------------------------------------------------------------------------------

\newpage
\begin{fullwidth}
~\vfill
\thispagestyle{empty}
\setlength{\parindent}{0pt}
\setlength{\parskip}{\baselineskip}
Copyright \copyright\ \the\year\ \thanklessauthor

\par\smallcaps{Published by \thanklesspublisher}

\par\smallcaps{\url{http://www.princeton.edu/manning3}}

\par License information.\index{license}

\par\textit{First printing, \monthyear}
\end{fullwidth}

%----------------------------------------------------------------------------------------

\tableofcontents % Print the table of contents

%----------------------------------------------------------------------------------------

%\listoffigures % Print a list of figures

%----------------------------------------------------------------------------------------

%\listoftables % Print a list of tables

%----------------------------------------------------------------------------------------
%	DEDICATION PAGE
%----------------------------------------------------------------------------------------

% \cleardoublepage
% ~\vfill
% \begin{doublespace}
% \noindent\fontsize{18}{22}\selectfont\itshape
% \nohyphenation
% Dedicated to my family and friends.
% \end{doublespace}
% \vfill
% \vfill

%----------------------------------------------------------------------------------------
%	INTRODUCTION
%----------------------------------------------------------------------------------------

\cleardoublepage
\chapter{Introduction}\label{ch:intro} % Adding an asterisk leaves out this chapter from the table of contents
This lab manual is intended to serve two purposes.  First, the manual
provides a comprehensive overview of official lab policies,
expectations, facilities, and personnel.  Second, it provides a set of
general tutorials and a list of relevant links, pointers, and/or
references related to the techniques we employ in our research.

\newthought{Who is this lab manual for?}  \marginnote{\texttt{TASK:}
  Upon reading through this lab manual for the first time, please make
  at least one edit.  You could correct a typo, clarify something
  that's unclear, add a comment, etc.  Focus your edits or additions
  on sections that are most relevant to the work you want to do.
  Importantly, be sure to push your edits to the manual's github
  repository so that everyone can benefit.} Every new lab member
should read the latest version of this lab manual in detail and
reference it later as needed.  Periodically throughout the document,
you will see margin notes with listed \texttt{TASK} items.  Completing
your read through entails both reading the contents of the manual and
completing the relevant \texttt{TASK} items.

\newthought{What should you do if you don't understand something?}
\marginnote{\texttt{TASK:} If you don't understand something, ask
  another lab member for help!}  If you don't understand something you
read in this manual, it is important that you \textit{ask another lab
  member for help}.  Every member of the lab brings their own unique
knowledge base, training, life experiences, and perspectives.
Respecting and celebrating those differences drives the science we do.
If you're new to the lab or new to a particular technique, you might
feel like a newbie today--- but chances are good that if you stick
around for a bit someone else will be seeking your expert opinion
before you know it.  In addition to learning, there's another good
reason for asking for help: if you don't understand something you read
in this manual, there's a reasonable chance that you've discovered a
mistake!

\newthought{Why is it worth my time to read through the manual?}
Aside from pursuing your own curiosity, a major reason that you've
decided to join an academic research lab is probably because you want
to gain training or career-advancing experiences.  This manual
summarizes the collective wisdom of past and present lab members in a
way that we think will best allow you to achieve your objectives.
\textit{Learn from it}, \textit{challenge it}, and \textit{add to it}.



\chapter{Official lab policies}\label{ch:policy}
Our lab's policies are intended to provide a framework for
\textit{maximizing efficiency}.  Achieving our peak efficiency as a
lab means we are being as scientifically productive as possible, in terms of knowledge
discovery (learning new stuff) and disemination (papers, talks,
conference presentations, publicly released datasets, software,
etc.). It also means that our fellow lab members are achieving their
training and career objectives.  To achieve peak efficiency we need to
succeed on two fronts:
\begin{itemize}
\item \textbf{Communication.}  We want to foster an environment where
  everyone feels comfortable contributing to the collective dialogue.
  Our lab meets regularly to discuss logistical (e.g.\ temporal, financial,
  sociological) and technical issues.  We also use a variety of
  software packages to synchronize and facilitate communication within
  our lab and between our lab and the broader scientific community.
\item \textbf{Resource allocation.}  Our lab resources (e.g.\
  equipment, time, money, attention) are finite.  We want to foster an
  environment where lab resources are used as efficiently as possible
  to achieve our collective goals.
\end{itemize}
Your job as a contributing lab member is to help us to achieve our 
collective peak efficiency (as a lab) while also maximizing your own training and
career potential.

\section{The lab hierarchy}
Our lab is organized in a roughly hierarchical structure.  Each lab
member's position in this hierarchy is determined by two factors: the
lab member's job title and the amount of time they have worked in the
lab.  Moving up in the hierarchy generally entails working in the lab
for some amount of time, gaining experience by working in other labs,
and/or earning academic degrees.  The lab hierarchy is intended to
serve as a general framework for estimating what is expected of each
lab member (e.g.\ in terms of research, supervising and mentoring,
training roles, and other lab responsibilities).  The hierarchy also
serves as a general framework for determining lab member salaries and
benefits.  The levels of the lab hierarchy are defined as follows:

\newcounter{levels}
\begin{list}{L\arabic{levels}-$x$:~}{\usecounter{levels}}
\item \textbf{Undergraduate research assistants.}  This category
  includes undergraduate students (currently enrolled at \ourschool)
  who are pursing an active undergraduate research program in the lab.
  An active undergraduate research program may include for-credit
  projects (such as an independent study or an honors thesis project)
  or not-for-credit projects.
\item \textbf{Postbaccalaureate research assistants.}  This category
  includes lab members who have already earned an undergraduate degree
  (e.g.\ BA, BS) but who have not earned a graduate degree, and who
  are not currently enrolled in a degree-granting graduate program.
\item \textbf{Postgraduate research assistants.}  This category
  includes lab members who have already earned a non-doctorate
  graduate degree (e.g.\ MA, MS) but who are not currently enrolled in
  a degree-granting graduate program.
\item \textbf{Graduate students.}  This category includes lab memebers
  who are currently enrolled in a degree-granting graduate program
  (generally working towards a master's degree or doctorate).
\item \textbf{Postdoctoral researchers.}  This category includes lab
  members who have earned a doctorate degree and are not currently
  enrolled in a degree-granting program.
\item \textbf{Principle investigator.}  This category includes lab
  members who have successfully obtained external funding for an
  independent research or training project, and whose funding is
  currently active.
\item \textbf{Lab director.}  \textit{There can be only one...}
\end{list}
Note that the ``$x$'' should be replaced with the time elapsed since
you joined the lab, in years.  In addition to the primary levels
enumerated above, certain support roles in the lab
exist outside of the main lab hierarchy:
\newcounter{supportlevels}
\begin{list}{S\arabic{supportlevels}-$x$:~}{\usecounter{supportlevels}}
\item \textbf{Administrative support staff.}  This category includes
  lab members who, regardless of their academic degree, are hired
  primarily to provide administrative assistance (e.g.\ assisting with
  grant or paper submissions, registration, scheduling, coordination)
  to facilitate scientific research in the lab.
\item \textbf{Research specialists.}  This category includes lab
  members who, regardless of their academic degree, are hired to bring
  a specific special scientific skill to the lab (e.g.\ programmers, graphic
  artists).
\end{list}

\section{Project hierarchies}
Lab members at any level may play a lead role on particular projects.
The role you play on a given project determines (for that project) your research and administrative
responsibilities, your supervisory responsibilities, your role in
decision making or strategizing, and your authorship position in papers or
conference presentations about the project.  Generally speaking, the
roles assigned to 

% Generally speaking, your immediate
% supervisor on a given project will be someone at a higher level of the
% hierarchy than you (and probably L4 or higher).  If you are looking
% for help with a particular technique, your best bet is to find someone
% who has been in the lab longer than you have, and who 

\section{Starting a new project}
% create bitbucket repo with the following folders (invite team
% members to have write access to repo; project lead and PI should have admin
% access):

% code/main -- code for running experiments and analyses.  this folder will eventually
% be made publically available (e.g. when a paper is published).

% code/dev -- under-development and/or half-baked code should go
% here.  all code starts here, and is moved to code/main after proper
% debugging and unit tests are written.

% code/debug -- unit tests 

% docs/papers -- each paper should exist as a
% subfolder of this directory.  each paper should also include a
% subfolder called "replicate" with links to data and code, along with
% *detailed* helpful instructions that would allow someone to download
% the repository and data and reproduce figures from each paper.  the
% document must list system requirements and dependencies, along with
% installation instructions for any required packages.  the directory
% should also include a "make_figs" script for each paper and poster;
% running the script should reproduce figures from the relevant
% paper/poster.  the folder should also include a spreadsheet (.csv)
% indicating which make_figs script goes with which paper/poster.  

%docs/posters -- each poster should
% exist as a subfolder of this directory 

%  also create a README file (in docs) containing a high-level
%  description of the project and an index describing the project
%  folder organization

%create shared evernote notebook; invite team members

%create redbooth project; invite team members to project

%project meeting (scheduled via doodle) with all relevant team members to create (project
%lead should create drafts of these documents prior to meeting):
%  roles spreadsheet
%  proposed budget spreadsheet
%  proposed timeline spreadsheet
%  



\section{Scheduling}
Complex dynamic systems can be difficult to understand (e.g.\
describe, compute with).  Fortunately for us, we do not need to start
entirely from scratch with respect to attempting to organize some
complex dynamic system we care about in our lab.  For example, we can
use tools like calendars and other software packages to organize and
understand our own temporal dynamics.  Our lab's scheduling policies
are intended to facilitate lab member interactions between ourselves,
our collaborators, and our experimental participants.

 \subsection{Attendance policy}
 As you move up in the lab hierarchy, our policy is to afford
 you increasing scheduling flexibility (which, in turn, assumes
 increasing responsibility on your part).  Increased
 \textit{scheduling flexibility} comes in the form of less frequent
 check-ins (e.g.\ times you are required to meet with your supervisor)
 and less structured research time (e.g.\ your level of independence
 as a researcher, as determined by your supervisor).  Increased
 \textit{responsibility} comes in the form of increased expectations
 placed on you as a researcher (in terms of research effort and
 productivity).




 \subsection{Lab calendar}
 \subsection{Scheduling lab events using Doodle}
 \subsection{Mandatory lab events}
 \subsection{Optional lab events}
 \subsection{Scheduling participants}

%\section{Research expectations}

% \section{Lab finances}
% \subsection{Equipment policy}
% \subsubsection{Computers}
% \subsubsection{Other research equipment}

% \subsection{Travel policy}

% \subsection{Poster printing}

% \subsection{Publication costs}

% \subsection{Subject payments}


% \subsection{Applying for research and training grants}


% \section{Lab mailing list and contact information}

% \section{Task management using Redbooth}

% \section{Code and document management using Bitbucket}

% \section{Data repository}
% \subsection{Lab Dropbox account}

% \section{Internal Review Board (IRB) approval process}
% \subsection{List of active protocols}
% \subsection{List of inactive protocols}


% \chapter{Official lab techniques}\label{ch:techniques}
% \section{Document formatting using \latex}
% \section{Coding experiments using PsychToolbox}
% \section{Data analysis using MATLAB and Python}



% \section{Behavioral experiments}

% \section{fMRI}
% \subsection{Running fMRI participants}
% \subsection{Data preprocessing}
% \subsection{Multivariate Pattern Analysis}
% \subsection{Topographic Factor Analysis}

% \section{Scalp EEG and ECoG}
% \subsection{Running EEG participants}

% \subsection{Running hospital patients (ECoG)}

% \section{Cluster computing}


% Citation example \cite{Tufte2001}, notice how the citation is in the margin. This is an example of how to add something to the index at the end of the document.\index{citation}

% \newthought{Example of} the \texttt{newthought} command for starting new sections. Typography examples: \allcaps{all caps} and \smallcaps{small caps}.

%------------------------------------------------

% \section{Figures}

% \lipsum[1] 

% \begin{marginfigure}
% \includegraphics[width=\linewidth]{helix}
% \caption{This is a margin figure. The helix is defined by $x = \cos(2\pi z)$, $y = \sin(2\pi z)$, and $z = [0, 2.7]$. The figure was drawn using Asymptote (\url{http://asymptote.sf.net/}).}
% \label{fig:marginfig}
% \end{marginfigure}

% \lipsum[2]

% \begin{figure*}[h]
% \includegraphics[width=\linewidth]{sine.pdf}
% \caption{This graph shows $y = \sin x$ from about $x = [-10, 10]$.
% \emph{Notice that this figure takes up the full page width.}}
% \label{fig:fullfig}
% \end{figure*}

% \lipsum[3]

% %------------------------------------------------

% \section{Tables} \marginnote{This is a random margin note. Notice that there isn't a number preceding the note, and there is no number in the main text where this note was written. Use \texttt{sidenote} to use a number.}

% \lipsum[4]

% \begin{table} % Add the following just after the closing bracket on this line to specify a position for the table on the page: [h], [t], [b] or [p] - these mean: here, top, bottom and on a separate page, respectively
% \centering % Centers the table on the page, comment out to left-justify
% \begin{tabular}{l c c c c c} % The final bracket specifies the number of columns in the table along with left and right borders which are specified using vertical bars (|); each column can be left, right or center-justified using l, r or c. To specify a precise width, use p{width}, e.g. p{5cm}
% \toprule % Top horizontal line
% & \multicolumn{5}{c}{Growth Media} \\ % Amalgamating several columns into one cell is done using the \multicolumn command as seen on this line
% \cmidrule(l){2-6} % Horizontal line spanning less than the full width of the table - you can add (r) or (l) just before the opening curly bracket to shorten the rule on the left or right side
% Strain & 1 & 2 & 3 & 4 & 5\\ % Column names row
% \midrule % In-table horizontal line
% GDS1002 & 0.962 & 0.821 & 0.356 & 0.682 & 0.801\\ % Content row 1
% NWN652 & 0.981 & 0.891 & 0.527 & 0.574 & 0.984\\ % Content row 2
% PPD234 & 0.915 & 0.936 & 0.491 & 0.276 & 0.965\\ % Content row 3
% JSB126 & 0.828 & 0.827 & 0.528 & 0.518 & 0.926\\ % Content row 4
% JSB724 & 0.916 & 0.933 & 0.482 & 0.644 & 0.937\\ % Content row 5
% \midrule % In-table horizontal line
% \midrule % In-table horizontal line
% Average Rate & 0.920 & 0.882 & 0.477 & 0.539 & 0.923\\ % Summary/total row
% \bottomrule % Bottom horizontal line
% \end{tabular}
% \caption{Table caption text} % Table caption, can be commented out if no caption is required
% \label{tab:template} % A label for referencing this table elsewhere, references are used in text as \ref{label}
% \end{table}

%----------------------------------------------------------------------------------------

\mainmatter

%----------------------------------------------------------------------------------------
%	CHAPTER 1
%----------------------------------------------------------------------------------------

%\chapter{Chapter 1 Title}
%\label{ch:1}

%------------------------------------------------

% \section{Section 1 - Fullwidth Environment Example}

% \begin{fullwidth}
% \lipsum[5]
% \end{fullwidth}

% \subsection{Subsection 1}

% \lipsum[6-7]

% \subsection{Subsection 2}

% \lipsum[7-8]



%----------------------------------------------------------------------------------------

\backmatter

%----------------------------------------------------------------------------------------
%	BIBLIOGRAPHY
%----------------------------------------------------------------------------------------

\bibliography{bibliography} % Use the bibliography.bib file for the bibliography
\bibliographystyle{plainnat} % Use the plainnat style of referencing

%----------------------------------------------------------------------------------------

\printindex % Print the index at the very end of the document

\end{document}