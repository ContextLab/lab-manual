%%%%%%%%%%%%%%%%%%%%%%%%%%%%%%%%%%%%%%%%%
% Tufte-Style Book (Minimal Template)
% LaTeX Template
% Version 1.0 (5/1/13)
%
% This template has been downloaded from:
% http://www.LaTeXTemplates.com
%
% License:
% CC BY-NC-SA 3.0 (http://creativecommons.org/licenses/by-nc-sa/3.0/)
%
% IMPORTANT NOTE:
% In addition to running BibTeX to compile the reference list from the .bib
% file, you will need to run MakeIndex to compile the index at the end of the
% document.
%
%%%%%%%%%%%%%%%%%%%%%%%%%%%%%%%%%%%%%%%%%

%----------------------------------------------------------------------------------------
%	PACKAGES AND OTHER DOCUMENT CONFIGURATIONS
%----------------------------------------------------------------------------------------

\documentclass[openany]{tufte-book} % Use the tufte-book class which in turn
                           % uses the tufte-common class

\definecolor{dartmouthgreen}{RGB}{0, 112, 60}

\hypersetup{colorlinks=true,linkcolor=dartmouthgreen} % Comment this line if you don't wish to have colored links

\usepackage{microtype} % Improves character and word spacing

%\usepackage{lipsum} % Inserts dummy text

\usepackage{booktabs} % Better horizontal rules in tables

\usepackage{graphicx} % Needed to insert images into the document
\graphicspath{{graphics/}} % Sets the default location of pictures
\setkeys{Gin}{width=\linewidth,totalheight=\textheight,keepaspectratio}
% Improves figure scaling

\usepackage[export]{adjustbox}

\usepackage{fancyvrb} % Allows customization of verbatim environments
\fvset{fontsize=\normalsize} % The font size of all verbatim text can be changed here

\newcommand{\hangp}[1]{\makebox[0pt][r]{(}#1\makebox[0pt][l]{)}} % New command to create parentheses around text in tables which take up no horizontal space - this improves column spacing
\newcommand{\hangstar}{\makebox[0pt][l]{*}} % New command to create asterisks in tables which take up no horizontal space - this improves column spacing

\usepackage{xspace} % Used for printing a trailing space better than
                    % using a tilde (~) using the \xspace command

\usepackage{hyperref} %web links/URLs

\usepackage{enumitem,amssymb}
\newlist{todolist}{itemize}{2}
\setlist[todolist]{label=$\square$}

\newcommand{\monthyear}{\ifcase\month\or January\or February\or March\or April\or May\or June\or July\or August\or September\or October\or November\or December\fi,\space\number\year} % A command to print the current month and year

\newcommand{\openepigraph}[2]{ % This block sets up a command for printing an epigraph with 2 arguments - the quote and the author
\begin{fullwidth}
\sffamily%\large
\begin{doublespace}
\noindent\allcaps{#1}\\ % The quote
\noindent\allcaps{#2} % The author
\end{doublespace}
\end{fullwidth}
}

\newcommand{\ourschool}{Dartmouth College}

\newcommand{\blankpage}{\newpage\hbox{}\thispagestyle{empty}\newpage} % Command to insert a blank page

\usepackage{makeidx} % Used to generate the index
\makeindex % Generate the index which is printed at the end of the document

%----------------------------------------------------------------------------------------
%	BOOK META-INFORMATION
%----------------------------------------------------------------------------------------

\title{Ask a question,\\ \noindent answer a question} % Title of the book


\author{Jeremy R. Manning, Ph.D.} % Author

\publisher{Contextual Dynamics Lab, \ourschool} % Publisher

%----------------------------------------------------------------------------------------

\begin{document}

\frontmatter

%----------------------------------------------------------------------------------------
%	EPIGRAPH
%----------------------------------------------------------------------------------------

\thispagestyle{empty}
%\openepigraph{Quotation 1}{Author, {\itshape Source}}
%\vfill
%\openepigraph{Quotation 2}{Author}
%\vfill
%\openepigraph{Quotation 3}{Author}

%----------------------------------------------------------------------------------------

\maketitle % Print the title page
%----------------------------------------------------------------------------------------
%	COPYRIGHT PAGE
%----------------------------------------------------------------------------------------

\newpage
\begin{fullwidth}
~\vfill
\thispagestyle{empty}
\setlength{\parindent}{0pt}
\setlength{\parskip}{\baselineskip}

\includegraphics[width=0.3in,left]{../lab_logo/CDL_Avatar_Cropped.png}\\\vspace{0.2in}

Copyright \copyright\ \the\year\ \thanklessauthor

\par\smallcaps{Published by the \thanklesspublisher}

\par\smallcaps{\url{http://www.context-lab.com}}

%\par License information.\index{license}

\par\textit{Current as of \monthyear}
\end{fullwidth}

%----------------------------------------------------------------------------------------

\setcounter{tocdepth}{1}
\tableofcontents % Print the table of contents

%----------------------------------------------------------------------------------------

%\listoffigures % Print a list of figures

%----------------------------------------------------------------------------------------

%\listoftables % Print a list of tables

%----------------------------------------------------------------------------------------
%	DEDICATION PAGE
%----------------------------------------------------------------------------------------

% \cleardoublepage
% ~\vfill
% \begin{doublespace}
% \noindent\fontsize{18}{22}\selectfont\itshape
% \nohyphenation
% Dedicated to my family and friends.
% \end{doublespace}
% \vfill
% \vfill

%----------------------------------------------------------------------------------------
%	INTRODUCTION
%----------------------------------------------------------------------------------------

\newcommand{\director}{Jeremy}
\newcommand{\coordinator}{Paxton}

\cleardoublepage
\chapter{Introduction}\label{ch:intro} % Adding an asterisk leaves out this chapter from the table of contents
This document contains a log of common issues and errors encountered by lab members.  If you encounter an issue you don't know how to resolve right away, this is a good place to begin your efforts. 
\\~\\
\noindent If you run into a tricky issue you think is likely to come up again, consider adding it to this document, and adding the solution when you resolve it.  Additionally, if you find yourself helping another lab member with a problem you encountered in the past, document the problem and its solution here for future lab members.


\chapter{Who to go to with questions}\label{ch: questions}
This section contains guidelines of where to direct questions related to technical and interpersonal issues that may arise, or any issues you encounter not already listed in the sections below. In general, resolving \textit{technical} issues within the lab saves time, but there will inevitably be some we cannot resolve on our own. In general, interpersonal issues should be resolved in accordance with your comfort, but it is important to be aware of the below resources should you encounter a situation in which you want to use them.

\subsection{Tech issues}
\begin{itemize}
\item For issues related to our lab's fileserver, seek assistance from \director.

\item For issues with other lab-owned equipment (e.g.~computers, research equipment, etc.) \coordinator~ or the senior lab member actively using the equipment should be your first-sought resource.

\item For issues withDartmouth IT, \href{help@dartmouth.edu}{open a ticket} with the Information, Technology, and Consulting office, or contact \href{mailto:Andrew.C.Knutsen@Dartmouth.edu}{Andrew Knutsen}.

\item For issues related to the the Discovery computing cluster, contact \href{mailto:John.P.Hudson@Dartmouth.edu}{John Hudson}.
\end{itemize}

\subsection{Lab documents}
\begin{itemize}
\item For questions about non-paper documents (e.g. receipts, consent forms, IRB protocols, the lab website, etc.), talk to \coordinator.

\item For questions on papers or posters, talk to \director.
\end{itemize}

\subsection{Lab policy}
\begin{itemize}
\item If you have a question about the lab's policies, first try to find the answer in the \href{https://github.com/ContextLab/lab-manual/tree/master/lab_manual.pdf}{Lab Manual}.

\item If you feel your question is not adequately answered in the Lab Manual, post your question on Slack (or, if it's a private issue, talk to \director).

\item If you think your question is likely to be of general interest, update the lab manual to reflect the answer.
\end{itemize}

\subsection{Specific methods}
\begin{itemize}
\item If you have questions about specific methods related to a project, ask a senior lab member working on the project.

\item It may also be helpful to check the \href{https://github.com/contextLab/cdl-tutorials}{CDL tutorials} repo, and contact the author(s) or contributors of a related tutorial. 

\item You can also post your question in the \href{https://context-lab.slack.com/messages/computrons/}{\#computrons channel} on Slack.
 
\item Remember, \href{https://www.google.com/}{Google} and \href{https://stackoverflow.com/}{Stack Overflow} are your best friends! You'll rarely ever encounter a problem someone else hasn't faced before.

\item For more general methods questions, talk to \coordinator.

\item If you still cannot find an answer to your question, talk to \director.
\end{itemize}

\subsection{Interpersonal issues}
\begin{itemize}
\item Clear, direct communication is often the best way to address interpersonal issues. However, if you are having trouble resolving something (or feel uncomfortable doing so on your own) you should reach out to one of the following resources:

\item Senior lab members (e.g. your immediate supervisor)

\item \director

\item The PBS department chair

\item The \href{https://students.dartmouth.edu/undergraduate-deans/}{Undergraduate Deans Office} or the \href{mailto:Graduate.and.Advanced.Studies@Dartmouth.edu}{Graduate and Professional Schools Deans Office}

\item Dartmouth's \href{https://www.dartmouth.edu/~hrs/}{Office of Human Resources}

\item Dartmouth's \href{https://www.dartmouth.edu/~eap/}{Faculty/Employee Assistance Program}

\item Dartmouth's \href{https://sexual-respect.dartmouth.edu/}{Title IX office} (for concerns about sexual misconduct, harassment, or assault)

\item \href{https://www.hanovernh.org/hanover-police-department}{Hanover Police} (if criminal activity is involved)
\end{itemize}

\chapter{Testing rooms}\label{ch: testrooms}
This section contains common issues and errors encountered when running subjects in our testing rooms. 


\chapter{Git \& GitHub}\label{ch: git}
This section contains common errors encountered using Git locally or syncing with the lab's GitHub repositories.


\chapter{Data analysis}\label{ch: analysis}
This section contains common errors encountered with many of the analysis-related software packages frequently used by the lab


\chapter{Computing cluster}\label{ch: cluster}
This section contains common errors encountered submitting and running jobs on Dartmouth's computing cluster.

\chapter{Other assorted issues}\label{ch: other}
This section contains commonly encountered issues that don't fit well in any of the above sections. Over time, if multiple issues begin to accumulate here that seem related to a broad topic, feel free to create a new section and move the related issues there.







% Citation example \cite{Tufte2001}, notice how the citation is in the margin. This is an example of how to add something to the index at the end of the document.\index{citation}

% \newthought{Example of} the \texttt{newthought} command for starting new sections. Typography examples: \allcaps{all caps} and \smallcaps{small caps}.

%------------------------------------------------

% \section{Figures}

% \lipsum[1] 

% \begin{marginfigure}
% \includegraphics[width=\linewidth]{helix}
% \caption{This is a margin figure. The helix is defined by $x = \cos(2\pi z)$, $y = \sin(2\pi z)$, and $z = [0, 2.7]$. The figure was drawn using Asymptote (\url{http://asymptote.sf.net/}).}
% \label{fig:marginfig}
% \end{marginfigure}

% \lipsum[2]

% \begin{figure*}[h]
% \includegraphics[width=\linewidth]{sine.pdf}
% \caption{This graph shows $y = \sin x$ from about $x = [-10, 10]$.
% \emph{Notice that this figure takes up the full page width.}}
% \label{fig:fullfig}
% \end{figure*}

% \lipsum[3]

% %------------------------------------------------

% \section{Tables} \marginnote{This is a random margin note. Notice that there isn't a number preceding the note, and there is no number in the main text where this note was written. Use \texttt{sidenote} to use a number.}

% \lipsum[4]

% \begin{table} % Add the following just after the closing bracket on this line to specify a position for the table on the page: [h], [t], [b] or [p] - these mean: here, top, bottom and on a separate page, respectively
% \centering % Centers the table on the page, comment out to left-justify
% \begin{tabular}{l c c c c c} % The final bracket specifies the number of columns in the table along with left and right borders which are specified using vertical bars (|); each column can be left, right or center-justified using l, r or c. To specify a precise width, use p{width}, e.g. p{5cm}
% \toprule % Top horizontal line
% & \multicolumn{5}{c}{Growth Media} \\ % Amalgamating several columns into one cell is done using the \multicolumn command as seen on this line
% \cmidrule(l){2-6} % Horizontal line spanning less than the full width of the table - you can add (r) or (l) just before the opening curly bracket to shorten the rule on the left or right side
% Strain & 1 & 2 & 3 & 4 & 5\\ % Column names row
% \midrule % In-table horizontal line
% GDS1002 & 0.962 & 0.821 & 0.356 & 0.682 & 0.801\\ % Content row 1
% NWN652 & 0.981 & 0.891 & 0.527 & 0.574 & 0.984\\ % Content row 2
% PPD234 & 0.915 & 0.936 & 0.491 & 0.276 & 0.965\\ % Content row 3
% JSB126 & 0.828 & 0.827 & 0.528 & 0.518 & 0.926\\ % Content row 4
% JSB724 & 0.916 & 0.933 & 0.482 & 0.644 & 0.937\\ % Content row 5
% \midrule % In-table horizontal line
% \midrule % In-table horizontal line
% Average Rate & 0.920 & 0.882 & 0.477 & 0.539 & 0.923\\ % Summary/total row
% \bottomrule % Bottom horizontal line
% \end{tabular}
% \caption{Table caption text} % Table caption, can be commented out if no caption is required
% \label{tab:template} % A label for referencing this table elsewhere, references are used in text as \ref{label}
% \end{table}

%----------------------------------------------------------------------------------------

\mainmatter

%----------------------------------------------------------------------------------------
%	CHAPTER 1
%----------------------------------------------------------------------------------------

%\chapter{Chapter 1 Title}
%\label{ch:1}

%------------------------------------------------

% \section{Section 1 - Fullwidth Environment Example}

% \begin{fullwidth}
% \lipsum[5]
% \end{fullwidth}

% \subsection{Subsection 1}

% \lipsum[6-7]

% \subsection{Subsection 2}

% \lipsum[7-8]



%----------------------------------------------------------------------------------------

\backmatter

%----------------------------------------------------------------------------------------
%	BIBLIOGRAPHY
%----------------------------------------------------------------------------------------

\bibliography{bibliography} % Use the bibliography.bib file for the bibliography
\bibliographystyle{plainnat} % Use the plainnat style of referencing

%----------------------------------------------------------------------------------------

\printindex % Print the index at the very end of the document

\end{document}