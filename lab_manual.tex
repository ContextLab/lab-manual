%%%%%%%%%%%%%%%%%%%%%%%%%%%%%%%%%%%%%%%%%
% Tufte-Style Book (Minimal Template)
% LaTeX Template
% Version 1.0 (5/1/13)
%
% This template has been downloaded from:
% http://www.LaTeXTemplates.com
%
% License:
% CC BY-NC-SA 3.0 (http://creativecommons.org/licenses/by-nc-sa/3.0/)
%
% IMPORTANT NOTE:
% In addition to running BibTeX to compile the reference list from the .bib
% file, you will need to run MakeIndex to compile the index at the end of the
% document.
%
%%%%%%%%%%%%%%%%%%%%%%%%%%%%%%%%%%%%%%%%%

%----------------------------------------------------------------------------------------
%	PACKAGES AND OTHER DOCUMENT CONFIGURATIONS
%----------------------------------------------------------------------------------------

\documentclass{tufte-book} % Use the tufte-book class which in turn
                           % uses the tufte-common class

\definecolor{dartmouthgreen}{RGB}{0, 112, 60}

\hypersetup{colorlinks=true,linkcolor=dartmouthgreen} % Comment this line if you don't wish to have colored links

\usepackage{microtype} % Improves character and word spacing

%\usepackage{lipsum} % Inserts dummy text

\usepackage{booktabs} % Better horizontal rules in tables

\usepackage{graphicx} % Needed to insert images into the document
\graphicspath{{graphics/}} % Sets the default location of pictures
\setkeys{Gin}{width=\linewidth,totalheight=\textheight,keepaspectratio} % Improves figure scaling

\usepackage{fancyvrb} % Allows customization of verbatim environments
\fvset{fontsize=\normalsize} % The font size of all verbatim text can be changed here

\newcommand{\hangp}[1]{\makebox[0pt][r]{(}#1\makebox[0pt][l]{)}} % New command to create parentheses around text in tables which take up no horizontal space - this improves column spacing
\newcommand{\hangstar}{\makebox[0pt][l]{*}} % New command to create asterisks in tables which take up no horizontal space - this improves column spacing

\usepackage{xspace} % Used for printing a trailing space better than
                    % using a tilde (~) using the \xspace command

\usepackage{hyperref} %web links/URLs

\newcommand{\monthyear}{\ifcase\month\or January\or February\or March\or April\or May\or June\or July\or August\or September\or October\or November\or December\fi\space\number\year} % A command to print the current month and year

\newcommand{\openepigraph}[2]{ % This block sets up a command for printing an epigraph with 2 arguments - the quote and the author
\begin{fullwidth}
\sffamily\large
\begin{doublespace}
\noindent\allcaps{#1}\\ % The quote
\noindent\allcaps{#2} % The author
\end{doublespace}
\end{fullwidth}
}

\newcommand{\ourschool}{Dartmouth College}

\newcommand{\blankpage}{\newpage\hbox{}\thispagestyle{empty}\newpage} % Command to insert a blank page

\usepackage{makeidx} % Used to generate the index
\makeindex % Generate the index which is printed at the end of the document

%----------------------------------------------------------------------------------------
%	BOOK META-INFORMATION
%----------------------------------------------------------------------------------------

\title{Lab Manual} % Title of the book

\author{Jeremy R. Manning, Ph.D.} % Author

\publisher{Contextual Dynamics Lab, \ourschool} % Publisher

%----------------------------------------------------------------------------------------

\begin{document}

\frontmatter

%----------------------------------------------------------------------------------------
%	EPIGRAPH
%----------------------------------------------------------------------------------------

\thispagestyle{empty}
%\openepigraph{Quotation 1}{Author, {\itshape Source}}
%\vfill
%\openepigraph{Quotation 2}{Author}
%\vfill
%\openepigraph{Quotation 3}{Author}

%----------------------------------------------------------------------------------------

\maketitle % Print the title page

%----------------------------------------------------------------------------------------
%	COPYRIGHT PAGE
%----------------------------------------------------------------------------------------

\newpage
\begin{fullwidth}
~\vfill
\thispagestyle{empty}
\setlength{\parindent}{0pt}
\setlength{\parskip}{\baselineskip}
Copyright \copyright\ \the\year\ \thanklessauthor

\par\smallcaps{Published by the \thanklesspublisher}

\par\smallcaps{\url{http://www.context-lab.com}}

%\par License information.\index{license}

\par\textit{First printing, \monthyear}
\end{fullwidth}

%----------------------------------------------------------------------------------------

\setcounter{tocdepth}{1}
\tableofcontents % Print the table of contents

%----------------------------------------------------------------------------------------

%\listoffigures % Print a list of figures

%----------------------------------------------------------------------------------------

%\listoftables % Print a list of tables

%----------------------------------------------------------------------------------------
%	DEDICATION PAGE
%----------------------------------------------------------------------------------------

% \cleardoublepage
% ~\vfill
% \begin{doublespace}
% \noindent\fontsize{18}{22}\selectfont\itshape
% \nohyphenation
% Dedicated to my family and friends.
% \end{doublespace}
% \vfill
% \vfill

%----------------------------------------------------------------------------------------
%	INTRODUCTION
%----------------------------------------------------------------------------------------

\newcommand{\director}{Jeremy}
\newcommand{\coordinator}{Kirsten}

\cleardoublepage
\chapter{Introduction}\label{ch:intro} % Adding an asterisk leaves out this chapter from the table of contents
This lab manual is intended to provide a crash course in doing
research in the Contextual Dynamics Lab.  It introduces our general research
approach and some basic lab policies.

\newthought{Who is this lab manual for?}  \marginnote{\texttt{TASK:}
  Upon reading through this lab manual for the first time, please make
  at least one edit.  You could correct a typo, clarify something
  that's unclear, add a comment, etc.  Importantly, be sure to push
  your edits to the manual's
  \href{https://bitbucket.org/manning3/lab-manual}{Bitbucket
    repository} so that everyone can benefit.}
Every new lab member should read
the latest version of this lab manual in detail and reference it later
as needed.  Periodically throughout the document, you will see margin
notes with listed \texttt{TASK} items.  Completing your read through
entails both reading the contents of the manual and completing the
relevant \texttt{TASK} items.

\newthought{What should you do if you don't understand something?}
\marginnote{\texttt{TASK:} If you haven't used \LaTeX~before (i.e.,
  the document formatting language in which this manual is written),
  you'll want to take a look at
  \href{https://www.latex-tutorial.com/tutorials/quick-start/}{this
    ``quick start'' tutorial}.}If you don't understand something you read in this
manual, it is important that you \textit{ask another lab member for
  help}.  Every member of the lab brings their own unique knowledge
base, training, life experiences, and perspectives.  Respecting and
celebrating those differences drives the science we do.  If you're new
to the lab or new to a particular technique, you might feel like a
newbie today--- but chances are good that if you stick around for a
bit someone else will be seeking your expert opinion before you know
it.  In addition to learning, there's another good reason for asking
for help: if you don't understand something, there's a reasonable
chance that you've discovered a mistake or a logical inconsistency!

\newthought{Why is it worth my time to read through the manual?}
Aside from pursuing your own curiosity, a major reason that you've
decided to join an academic research lab is probably because you want
to gain training or career-advancing experiences.  This manual briefly
summarizes the collective wisdom of past and present lab members in a
way that we think will best allow you to achieve your objectives.
\textit{Learn from it}, \textit{challenge it}, and \textit{add to it}.

\newthought{What ``isn't'' this lab manual?}
This lab manual is
\textit{not} intended to provide a comprehensive overview of
everything you need to know to do your research projects.  As
described next, you may not even \textit{know} what you need to know
to do your projects!  Nevertheless, you need somewhere to start, and
this is that place.




\chapter{Official lab practices and policies}\label{ch:policy}
Our lab's practices and policies are intended to provide a framework
for \textit{maximizing efficiency}.  Achieving our peak efficiency as
a lab means we are being as scientifically productive as possible, in
terms of knowledge discovery (learning new stuff) and dissemination
(papers, talks, conference presentations, publicly released datasets,
software, etc.). It also means that our fellow lab members are
achieving their training and career objectives.  To achieve peak
efficiency we need to succeed on three fronts:
\begin{itemize}
\item \textbf{Communication.}  We want to foster an environment where
  everyone feels comfortable contributing to the collective dialogue.
  Our lab meets regularly to discuss logistical (e.g.\ scheduling, financial,
  sociological) and technical issues.  We also use a variety of
  software packages to synchronize and facilitate communication within
  our lab and between our lab and the broader scientific community.
\item \textbf{Resource allocation.}  Our lab resources (e.g.\
  equipment, time, money, attention) are finite.  We want to foster an
  environment where lab resources are used as efficiently as possible
  to achieve our collective goals.  We also want to foster
  sustainable use of resources by regularly pursuing research funding opportunities.
\item \textbf{Adaptability.}  The whole point of \textit{research} is that we
  don't already know the answer to the questions we're exploring or
  how to make the tools we're working on.  That means that we won't
  necessarily be able to plan out everything in advance.  We often need to
  be focused and efficient \textit{without knowing the end goal}!
\end{itemize}
Your job as a contributing lab member is to help us to achieve our
collective peak efficiency (as a lab) while also maximizing your own
training and career potential.  To do this, the Contextual Dynamics
Lab practices \textbf{agile research}, as described in the next section.

%\newpage
\section{Doing agile research}
\marginnote{\texttt{TASK:} Watch the
  \href{http://scrumtrainingseries.com/}{Scrum Training Series videos}
  and answer the quiz questions as you go. (Note: ``Scrum'' is an implementation
  of the agile process.) In the lab we'll adapt this approach
  somewhat, but we will try to preserve the core ideas.  For example,
  you can mentally replace ``software'' with ``software, papers, and
  posters.''  Also, the meeting and team structures we use in the lab
  will be different from those used in a software development
  environment.  Again, we'll try to preserve the core principles.} The
agile approach to research we use in the Contextual Dynamics Lab is
inspired by the Agile Movement in the software development world.  The
idea is to create learning, adaptable teams to work on very small
bite-sized tasks.  Specifically, project teams are designed to respond
to unpredictability in research through incremental, iterative work
``sprints.''  Each sprint lasts approximately 1--2 weeks, and results
in a demonstrable research product (e.g.\ a draft of a paper, a draft
of a grant, a completed analysis or figure, a poster, a software tool,
etc.).

This is different from traditional approaches that you may have
encountered in other labs or work environments, where a research team
might try to plan out every part of a project in advance in a series
of small steps.  We still try to break projects into tiny bite-sized
chunks, but the key insight of the agile approach is that we only need
to know what the \textit{next} chunk is.  Although it's helpful to
have a general (if vague) sense of where things are going, we never
actually need to know where the project will ultimately end up.  The
goals and process are constantly evolving.  Perhaps the best
justification for this approach is that \textbf{the first day of a new
  research project is when you're the most clueless about what you'll
  find}.  So how could that possibly be the ideal time to plan out the
entire project?\marginnote{\texttt{NOTE:} Our adapted approach also
  draws inspiration from
  \href{http://technocalifornia.blogspot.com/2008/06/agile-research.html}{this
    blog}.}

Our \href{http://agilemanifesto.org/}{agile research manifesto} has three key
tenets:
\begin{enumerate}
\item Value \textbf{individuals and interactions} over
  \textit{processes and tools}.
\item Value \textbf{working, intuitive, tools and research products} over
  \textit{comprehensive documentation}.
\item Value \textbf{responding to change} over \textit{following a plan}.
\end{enumerate}
While there is value to the \textit{italicized} items, we value
\textbf{bolded} items more.  There are \href{http://www.agilemanifesto.org/principles.html}{twelve
  principles} we use to achieve these tenets:
\begin{enumerate}
\item Our highest priority is to benefit the research community
  through the early and continuous delivery of documentation (papers,
  presentations) and tools (software and devices).

\item Welcome changing requirements, even late in development.  Agile
  processes harness change to gain a competitive advantage.

\item Deliver working software and papers frequently, from a couple of
  weeks to a couple of months, with a preference to the shorter
  timescale.

\item Researchers in different roles must work together daily
  throughout the project.

\item Build projects around motivated individuals.  Give them the
  environment and support they need and trust them to get the job
  done.

\item The most effective and efficient method of conveying information
  to and within a research team is face-to-face conversation.

\item The product itself (software, paper, poster,
  presentation, grant) is the primary measure of progress.

\item Agile research is sustainable research.  Researchers should
  be able to maintain a constant pace indefinitely.

\item Continuous attention to technical excellence and good design
  enhances agility.

\item Simplicity---the art of maximizing the amount of work not
  done---is essential.

\item The best architectures, requirements, and designs emerge from
  self-organizing teams.

\item At regular intervals, the team reflects on how to become more
  effective, then tunes and adjusts its behavior accordingly.
\end{enumerate}





\subsection{Projects roles}
Every project has four possible roles.  You will play one or more of
these roles on your project:
\begin{enumerate}
\item \textbf{Project Owner.}  This is the person responsible for
  maximizing ``return on investment'' of the project effort:
\begin{enumerate}
\item Responsible for project vision
\item Constantly re-prioritizes the research backlog, adjusting any
  long-term expectations such as publication and release plans
\item Final arbiter of requirements questions
\item Accepts or rejects each project increment
\item Decides whether to publish/ship the project
\item Decides whether to continue development
\item Considers interests of funding bodies (e.g. NIH, NSF, DARPA,
  private organizations) and the scientific community
\item May contribute as a team member
\item Has a leadership role
\item Usually the project owner will be \director
\end{enumerate}

\item \textbf{Team Member.}  Team members are responsible for carrying
  out the project work.  Team members are:
\begin{enumerate}
\item Cross-functional: includes members with development skills
  (write code or papers/grants), testing skills (e.g.\ test software,
  proofread papers/grants), and/or domain expertise (e.g.\ knowledge
  or interest in a relevant research area)
\item Self-organizing and self-managing without externally assigned
  roles
\item Negotiates commitments with the Project Owner, one ``sprint'' at
  a time
\item Has autonomy regarding how to reach commitments
\item Intensely collaborative
\item (Ideally) located in one team room (usually this will be the lab)
\item (Ideally) committed to long-term, full-time lab membership
\item (Ideally) members of a single team/project
\item Has a leadership role
\end{enumerate}

\item \textbf{Project Coordinator.} The Project Coordinator facilitates
  the agile research process by:
\begin{enumerate} 
\item Helping to resolve impediments
\item Creating an environment conducive to team self-organization
\item Capturing empirical data to adjust forecasts
\item Shielding the team from external interference and distraction to
  keep it ``in the zone''
\item Enforcing timelines
\item Having no management authority over the team (anyone with authority
  over the team is by definition not its Project
  Coordinator)
\item Having a leadership role
\item Usually the Lab Coordinator (\href{mailto:Kirsten.K.Ziman@dartmouth.edu}{Kirsten Ziman}) will also be the Project Coordinator
\end{enumerate}

\item \textbf{Collaborator.}  Collaborators are not formally part of
  the project team, and generally will not attend regular meetings as
  part of the team.  Collaborators do not have a leadership role in
  the project.  They may carry out one or more of the
  following roles:
\begin{enumerate}
\item Provide data or share equipment
\item Provide occasional consulting services
\item Provide occasional feedback on project results
\item Carry out minor analyses
\item Proofread documents
\item Help with administrative tasks such as scheduling
\item Help with information technology tasks such as computer
  maintenance
\item A project may never be held up by a collaborator.  If the
  collaborator fails to provide a promised service, the project team
  must adapt.  If the collaborator fails to meet a non-critical
  deadline, the project will proceed without that component of the
  project.  Involvement as a collaborator is fluid.
\end{enumerate}
By definition, collaborators play a minor role in the project, and they are not
responsible for managing any aspect of the project.  They may become Team
Members if their involvement increases.  Generally collaborators will
be included in a paper's acknowledgement section, but collaborators
are not normally co-authors.
\end{enumerate}


\subsection{Project meetings}
Effective lab communication requires forums for communicating.  As
described below, we use \href{http://www.slack.com}{Slack} to
facilitate non-in-person communications, but Slack cannot replace
in-person meetings.  In fact, our approach is set up to encourage
in-person interactions as often as possible-- ideally at least once
per day.  We'll have the following regularly scheduled meetings:
\begin{enumerate}
\item \textbf{Lab meeting.}  We will have, as a lab, a regular weekly
  2 hour meeting where we discuss current progress, impediments to
  progress, and plans for the upcoming week.  We will also use the
  meeting time to update and prioritize the research
  backlog\marginnote{\texttt{TIP:} The research backlog is our list of project
    ideas.  It will grow and change over time.} and plan sprints for
  that week.  (In \href{http://scrumtrainingseries.com/}{Scrum}
  terminology, lab meeting will combine the notions of the
  \href{http://scrumtrainingseries.com/BacklogRefinementMeeting/BacklogRefinementMeeting.htm}{Backlog
    Refinement Meeting} and the
  \href{http://scrumtrainingseries.com/SprintPlanningMeeting/SprintPlanningMeeting.htm}{Sprint
    Planning Meeting}.)  Attendees: everyone in the lab.
\item \textbf{Daily team meeting.}  You and your fellow project
  members will meet at least once daily, for 15 minutes.  You'll
  schedule this meeting as a group based on what's most convenient.
  The idea is to discuss:
\begin{enumerate}
\item What each team member did since yesterday
\item What each team member plans to do today
\item Any impediments to progress
\end{enumerate}
Because of the content of this meeting, it's ideal to schedule the
meetings for early in the day.  This meeting is based on the notion of
a
\href{http://scrumtrainingseries.com/DailyScrumMeeting/DailyScrumMeeting.htm}{Daily
  Scrum Meeting}. Attendees: all team members (except the Project Owner
and Project Coordinator).
\item \textbf{Weekly team meeting.}  Each team will meet for 1 hour
  each week to do live demonstrations of working products for the
  Project Owner and any external collaborators.  This could be in the
  form of a software demonstration, a couple of slides demonstrating a
  figure, a poster (e.g. for a conference), or a draft of a document
  or section of a document.  The meeting is based on the notion of a
  \href{http://scrumtrainingseries.com/SprintReviewMeeting/SprintReviewMeeting.htm}{Sprint
    Review Meeting}.  Attendees: all team members (Project Coordinator
  attendance is optional).
\item \textbf{Tutorial meetings.}  Once every other week, we will meet
  for 1 hour to discuss a method or technique of general interest and
  relevance to the lab.  Usually this will entail a student-lead
  discussion of a journal article, software package, or technique.
  The presentations may be slide-based (e.g.\ PowerPoint, Keynote,
  Beamer) or whiteboard-based.  Presentations should be informal and
  heavily discussion oriented.  Attendees: optional, but all lab
  members and collaborators are encouraged to attend.
\end{enumerate}

%\newpage
\section{Getting started in the lab}
\marginnote{\texttt{TASK:} Create Slack, Trello, and BitBucket accounts.}The very first thing you need to do is to create three accounts that
will enable you to interact with the rest of the lab, download and use
the lab's software packages, and accomplish various necessary
administrative tasks:
\begin{enumerate}
\item \href{https://context-lab.slack.com}{\textbf{Slack.}}  This is where
  almost all not-in-person lab communications take place.  It provides
  an interface for asking questions, storing notes, and sharing
  ideas.  If you have an @dartmouth.edu email address, you can create
  an account without an invitation.  Otherwise you'll need to get invited by \coordinator.

\item \href{https://www.trello.com}{\textbf{Trello.}}
  \marginnote{\texttt{TASK:} If you've never used Trello before, you
    may find it useful to work through this
    \href{https://trello.com/b/I7TjiplA/trello-tutorial}{Trello
      tutorial}.}  This is used to manage all projects and tasks.  It
  provides a way of tracking progress and impediments to progress.
  You'll need to sign up for a free account and then have the Lab
  Coordinator invite you to the lab's Trello Team.

\item \href{https://www.bitbucket.org}{\textbf{BitBucket.}}
  \marginnote{\texttt{TASK:} If you've never used BitBucket (Git)
    before, please work through the
    \href{https://confluence.atlassian.com/bitbucket/bitbucket-tutorials-teams-in-space-training-ground-755338051.html}{BitBucket
      Tutorials}.}  This is used to manage all code, papers, grants,
  presentations, and posters.  In other words, anything where it'd be
  useful to track multiple versions, anything that we might ultimately
  want to release to the public, and/or anything that multiple lab
  members will be collaborating on.  Each project has one or more
  BitBucket repositories.  We also use
  \href{https://www.atlassian.com/software/sourcetree}{SourceTree} as
  a graphical interface for Bitbucket.
\end{enumerate}

Once you've created those accounts (and once you've been invited to
the lab-specific team on Trello), you can start working through the
\href{https://trello.com/b/Mgs45iEJ/lab-setup}{Lab Setup Trello
  Board}.  You can ask any questions through Slack (use the \href{https://context-lab.slack.com/messages/general/}{\#general
channel} or the channel specific your project).


\section{Starting a new project}
Our lab uses a number of project management tools and policies to
promote continuity across projects and lab members.  First, make sure
that your project doesn't already exist.  If you're sure your project
doesn't already exist, follow the steps
\href{https://trello.com/b/s4L8fbot/initiate-a-project}{here} to
initiate a new project.

%\newpage
\section{Joining a project}
To join a project, simply subscribe to the project's Slack channel and
update the
\href{https://trello.com/b/IWQ00r69/project-staffing-collaborators}{Project
  Staffing \& Collaborators} board on Trello.  All project
communications should occur through Slack, whenever possible.  This
keeps notes searchable and visible to all team members (except direct
messages, which are useful for private communications between one or
more team members).


\section{Scheduling}
Our lab's scheduling practices and policies are intended to facilitate
lab member interactions between ourselves, our collaborators, and our
experimental participants.  There are three basic tools the lab uses
to organize and schedule events:
\begin{itemize}
\item \href{http://calendar.google.com}{Google Calendar}.  We use the
  \href{https://calendar.google.com/calendar/ical/5ta50cfv4uih0a0k8m2di9dhjo\%40group.calendar.google.com/private-ff1338ddce84ac37d5ab682cd94e7f69/basic.ics}{main lab calendar} to keep track of lab-wide events including lab
  meetings, conferences, important talks.  We use the \href{https://calendar.google.com/calendar/ical/j6noo2tqahpsoq9na1h16paf3s\%40group.calendar.google.com/private-c3d75bea1ab4605947353d159d3dcd05/basic.ics}{DHMC events
  calendar} to keep track of important events and meetings at the
  DHMC.  We use the \href{https://calendar.google.com/calendar/ical/h1j06dohcg7v1g2o5tkb7ijhvs\%40group.calendar.google.com/private-239aaf8b4dc60480c90e8d7fc353e229/basic.ics}{out-of-lab calendar} to keep track of known
  absences (e.g. illness, travel, holidays,
  etc.).\marginnote{\texttt{TASK:} ask \coordinator~to send you
    an invite to the out-of-lab calendar so that you can add your
    planned absences.}  We use the
  \href{https://calendar.google.com/calendar/ical/dgcv8l8a8s10hfg2s5h0qec0q0\%40group.calendar.google.com/private-4810aed94f818d5748045447ab46c62d/basic.ics}{CDL
    resources calendar} to coordinate the use of shared rooms and
  equipment, such as testing rooms, our EEG system, hospital
  equipment, etc.  You may also choose to create project-specific
  Google Calendars, inviting project team members.  When you add an
  event (in any lab calendar), it is important to include
  the following information as a comment (this does not apply to ``out-of-lab'' events):
\begin{itemize}
  \item Key contact names and contact information (email or phone)
  \item Physical address (where the event will take place)
  \item A brief description of the event and/or other relevant information
  \item Attach any relevant documents via Google Docs
\end{itemize}
\item \href{http://www.doodle.com}{Doodle} and
  \href{http://www.when2meet.com}{When2Meet}.  We use Doodle and
  When2Meet to converge on mutually good meeting times that fit (as
  well as possible) with everyone's busy schedules.  Doodle is most
  useful for selecting a date from a large number of options, and
  When2Meet is most useful for selecting a specific time on a
  relatively small number of dates.
\end{itemize}

 \subsection{Attendance policy}
 In general, we expect full time employees to be in the lab during
 ``standard'' working hours-- roughly between 9 AM and 5 PM.  The
 precise range of hours you work is less important to us than putting
 in an effort to help form a cohesive lab culture where lab members
 can interact in person to share ideas, leverage expertise, solve
 problems, etc.  Therefore, even if you end up deciding to shift your
 hours, we'd like you to make a strong effort to be physically present
 in the lab between 1 and 4 pm (prior arrangements notwithstanding;
 e.g. if you have a long commute and we've agreed that you won't come
 in every day, and if you need to occasionally schedule an
 appointment during the 1--4 pm window you can do so).  Similarly, if
 you are a part time employee, we'd like you to try to put in your
 in-the-lab hours during the 1 to 4 pm time window as often as
 possible.

The lab also abides by Dartmouth's standard paid time off policies for
benefits-eligible (full-time, non-student) employees.  If you are a
salaried employee, you can find the official policy
\href{http://www.dartmouth.edu/~hrs/pdfs/paid_time_off_salaried.pdf}{here},
and if you are an hourly employee, you can find the official policy
\href{http://www.dartmouth.edu/~hrs/pdfs/Paid_Time_Off_Hourly.pdf}{here}.
If you are a student employee, you are generally ineligible for paid
time off (you can take time off, but you won't normally be paid for it).

If you know that you'll be unable to meet any of these general attendance
guidelines, please coordinate with \director~to make appropriate
arrangements.  With the above in mind, we abide by a ``common sense''
attendance policy that relies on an honor
system.\marginnote{\texttt{TIP:} If you are an hourly employee, you'll
  need to track your hours using the Kronos system as described on the
  \href{https://trello.com/b/Mgs45iEJ/lab-setup}{Lab Setup Trello
    Board}.}  If you cannot attend a lab event or meeting, your
privacy will be respected: you do not (generally speaking) need to
provide a reason for your absence (although you are honor bound not to
abuse this system!)-- but you are expected to responsibly manage your
schedule so that you get your work done and minimize inconvenience to
others to the extent possible.  The one exception is that if you seem
to be abusing this system (e.g.\ as determined by your project owner,
project coordinator, or fellow team members), you may be asked to
provide additional information (in a way that does not invade your
privacy-- and if you are worried that this policy is overly intrusive,
please bring your concerns to \director~or \coordinator).  Here's the
official lab attendance policy:
\begin{itemize}
\item It is your responsibility to provide notice, well in advance, to anyone your absence will
  affect (e.g. \director, \coordinator, people you're scheduled to
  meet with, etc.).  The best way to do this is via email or Slack.
  We also use an
  \href{https://calendar.google.com/calendar/ical/h1j06dohcg7v1g2o5tkb7ijhvs\%40group.calendar.google.com/private-239aaf8b4dc60480c90e8d7fc353e229/basic.ics}{``out-of-lab''
    calendar} to keep track of things like vacations or longer planned
  absences.
\begin{itemize}
\item You are responsible for accounting for your planned absences
  when we plan our weekly sprints.  If you agree to take on work
  during a sprint, you're responsible for it until you make
  alternative plans with your team!
\item Prolonged (more than 1 day, excluding weekends and
  lab-related absences) planned
  absences should be scheduled at least 1 week in advance, and ideally
  2 weeks in advance.
\item Brief (one day) absences (excluding weekends, and lab-related
  absences) should be scheduled as far in advance as possible, but at
  least at the beginning of the week.
\end{itemize}

\item If you are feeling sick, \textit{do not come into the lab}.  We can
  arrange virtual meetings (if you are feeling well enough) or
  re-schedule as needed.  The health and safety of the lab is the top
  priority.

\item If you need to be out of the lab for an unexpected non-illness-related
  emergency, simply give as much notice and information as possible.

\item You are expected to attend all lab meetings and other regularly
  scheduled meetings that are directly relevant to your work in the
  lab.

\item If you are scheduled to present at a conference (i.e.\ you
  submit an abstract and the abstract is accepted as a talk or
  poster), you are expected to attend the conference to present your
  work.

\item You are strongly encouraged (but never required) to attend
  relevant journal clubs, PBS talks, DHMC
  meetings and talks, thesis defenses, and other relevant lab and/or
  Dartmouth-affiliated events.  If you are overwhelmed with other
  work, have a conflicting meeting, are running an experimental
  participant, or are out of the lab for other reasons, you do not
  need to provide a reason for your absence (unless you're presenting
  or are otherwise playing a key role).
\end{itemize}

\subsection{Compensation and benefits}
If you are a non-student full-time on-campus employee, it's likely that
you're eligible for Dartmouth benefits, such as medical insurance,
dental insurance, life insurance, etc.  You can read more about the
comprehensive benefits package \href{http://www.dartmouth.edu/~hrs/benefits/}{here}.

Dartmouth also sponsors various health-benefits programs (for all
members of the Dartmouth community).  For example, you are likely
eligible to get a free (or subsidized) fitness tracker, fitness
equipment, race fees, gym membership, etc.  You can also earn cash (up
to \$400/year) for meeting your fitness goals.  Go
\href{http://join.virginpulse.com/dartmouth/}{here} to learn more or
sign up for this program.

If you are a student employee, you may be paid or unpaid.  In general,
full-time student employees are paid and part-time student employees
are unpaid until they have been a full-time employee for at least one
term.  Your precise level of compensation will depend on your
position, how your work in the lab is funded, your prior research
experience, etc.

\subsection{Interpersonal issues}
All lab members, regardless of position or status, are protected by
(and must abide by) Dartmouth's human resources policies.  This means
behaving professionally and respectfully towards others (including,
but not limited to, your fellow lab members).  On (hopefully) rare
occasions, despite your best efforts, you may find yourself in an
interpersonal situation that you feel unable to resolve on your own.
You have many resources at your disposal to help get you back on
track.

The \href{http://www.dartmouth.edu/~hrs/}{Office of Human Resources}
provides assistance and resources to all faculty, staff, retirees, and
prospective employees. The
\href{http://www.dartmouth.edu/~seo/}{Student Employment Office}
provides a similar suite of services to student employees.  The
\href{http://www.dartmouth.edu/~ombuds/}{Dartmouth Ombuds Office} also
provides confidential and informal assistance in resolving concerns
related to interpersonal issues.  If you are experiencing an
interpersonal issue with another lab member and are having trouble
resolving it on your own, please seek out
assistance from \director, \coordinator, your Project Owner, your
Project Coordinator, or one of the above Dartmouth community resources
as early as possible.


 \section{Lab resources}
 As with most academic research labs, we (sadly!) must conduct our
 research within a limited research budget.  In practice, the
 important thing is to communicate with \director~and/or the Lab
 Coordinator before you spend (or commit to spending) lab funds.

 Generally, the lab's financial policy is the following: we will do
 whatever is possible to ensure you have the equipment and resources
 you \textit{need} to do your best work.  If you can adequately
 justify an expense and sufficient funds are available, then we will
 spend what it takes to get the job done.  If you cannot justify an
 expense, or if the lab does not have sufficient funds, then we will
 need to get creative by figuring out how to get the job done anyway
 on a seemingly too-small budget.  Usually we'll find ourselves
 somewhere in the middle of this continuum, which will help us to
 stretch our limited budget as far as possible while not making
 ourselves crazy or losing too much productivity in the process.

 Some of our projects are intended to be self-funded and/or to support
 other projects (e.g.\ StockProphet).  Any use of project-generated
 funds should be discussed with \director~and/or \coordinator.

 \subsection{Computers}
All lab members need a computer to get their work done.  We generally
prefer to use Macs, as this maximizes compatibility across lab
members.  Depending on your expected role in the lab and the specifics
of your project, the lab may provide a computer to you, or you may be
expected to use your personal computer to complete your work.  Any
equipment purchased by the lab, including personal computers, is the
official property of the Contextual Dynamics Lab and should be treated
as such.  All equipment must be returned to the lab when your
association with the lab is complete.

In addition to personal computers, we also maintain a lab account on
Dartmouth's \href{http://techdoc.dartmouth.edu/discovery/}{Discovery
  Supercomputing Cluster}.  In addition to having access to the
compute nodes shared amongst the entire Dartmouth community, we have
purchased several dedicated servers and a powerful head node that is
shared with the \href{http://www.cosanlab.com/}{COSAN Lab} and the
\href{http://www.dartmouth.edu/~bil/}{Dartmouth Brain Imaging Lab}.
When you link your Discovery account with the lab, you will
automatically have access to those additional computing resources.  We
use Discovery for our most computationally intensive work.  The
\href{https://trello.com/b/Mgs45iEJ/lab-setup}{Lab Setup Trello Board}
contains instructions for creating an account and accessing the
Discovery cluster.  Our suggested workflow is to do non-intensive
computations and analyses on your personal desktop or laptop computer,
and to offload more intensive analyses to Discovery.  The lab's code
repository includes sample MATLAB scripts for submitting analyses to
Discovery.

 \subsection{Other research equipment}
Many research projects require specialized research equipment
(e.g.\ for neuroimaging using fMRI, EEG, ECoG, etc.).  Some of the
necessary research equipment is owned by the Contextual Dynamics Lab,
and other equipment is shared with other labs or with PBS.  All
equipment should be treated with care and respect.  Any malfunctions
should be reported immediately.

 \subsection{Travel policy}
 A major component of doing scientific research is communicating with
 other scientists.  The Contextual Dynamics Lab regularly presents at
 several international scientific conferences.  If you are presenting
 your work from the lab (i.e., you are the presenting author for a
 talk or poster), then your travel expenses and conference
 registration fees will be guaranteed by the lab, under the assumption
 that you will also make reasonable efforts to seek out alternative
 sources of travel funding (e.g.\ through PBS, other internal
 Dartmouth sources, apply for travel awards, use personal grants like
 NRSAs or NSF fellowships, etc.).  You are also expected to keep costs
 low (e.g.\ fly economy class, seek out cheaper tickets, stay in
 reasonably priced hotels, share a room with other lab members, etc.).
 By the same token, we also want to be cognizant of your comfort and
 time, and it is not always necessary to use the cheapest option.
 More specific travel guidelines will be given on a per-conference or
 per-trip basis.

 If you are not presenting your work (or if you're presenting non-lab
 work), but you are a full-time graduate student, postdoctoral
 researcher, or lab coordinator, then the lab may cover your travel
 expenses to a limited number of conferences each year.  These should
 be discussed with \director.

If you are an undergraduate research not presenting your work, the lab
will generally not pay for you to attend conferences.  However, if you
are interested in attending a conference, and you aren't able to
secure funding through non-lab sources, you should discuss your
options with \director.

 \subsection{Poster printing}
 There are two on-campus poster printers.  One is in the Map Room of
 Baker Library.  More information may be found
 \href{http://www.dartmouth.edu/~library/maproom/printingfaq.html}{here}.
 The Map Room printer should be used in most cases.  The other printer
 maintained through the Thayer School of Engineering.  More
 information may be found
 \href{http://kb.thayer.dartmouth.edu/article/286-printing}{here}.  It
 is important to schedule your printing time as far in advance as
 possible, particularly before conferences when many people will want
 to print.  Advanced planning can help us avoid the additional costs
 associated with off-campus printers.


 \subsection{Publication costs}
All costs related to lab publications will be fully covered by the
lab.  \coordinator~can help facilitate these payments.

 \subsection{Subject payments}
 Subject payments for lab research projects will be fully covered by
 the lab.  Subject payment guidelines are generally found in the IRB
 approval documentation relevant to your project.  For Mechanical Turk
 experiments, a subject payments budget should be approved prior to
 beginning the experiment.


\chapter{Internal Review Board (IRB) approvals}
Experimental \marginnote{\texttt{TASK:} prior to running any out-of-lab
  participants in your study, you must (a) verify that your experiment has
  been approved by the IRB and (b) verify that your name is
  specifically listed on the associated protocol.} protocols and IRB approval forms are maintained and
coordinated through our
\href{https://context-lab.slack.com/archives/admin}{dedicated Slack
  channel}.

 \section{List of active protocols}
The lab does not currently have any active protocols.  When protocols
are approved, they will be added here.

 \section{List of inactive protocols}
The lab does not have any inactive protocols.



% Citation example \cite{Tufte2001}, notice how the citation is in the margin. This is an example of how to add something to the index at the end of the document.\index{citation}

% \newthought{Example of} the \texttt{newthought} command for starting new sections. Typography examples: \allcaps{all caps} and \smallcaps{small caps}.

%------------------------------------------------

% \section{Figures}

% \lipsum[1] 

% \begin{marginfigure}
% \includegraphics[width=\linewidth]{helix}
% \caption{This is a margin figure. The helix is defined by $x = \cos(2\pi z)$, $y = \sin(2\pi z)$, and $z = [0, 2.7]$. The figure was drawn using Asymptote (\url{http://asymptote.sf.net/}).}
% \label{fig:marginfig}
% \end{marginfigure}

% \lipsum[2]

% \begin{figure*}[h]
% \includegraphics[width=\linewidth]{sine.pdf}
% \caption{This graph shows $y = \sin x$ from about $x = [-10, 10]$.
% \emph{Notice that this figure takes up the full page width.}}
% \label{fig:fullfig}
% \end{figure*}

% \lipsum[3]

% %------------------------------------------------

% \section{Tables} \marginnote{This is a random margin note. Notice that there isn't a number preceding the note, and there is no number in the main text where this note was written. Use \texttt{sidenote} to use a number.}

% \lipsum[4]

% \begin{table} % Add the following just after the closing bracket on this line to specify a position for the table on the page: [h], [t], [b] or [p] - these mean: here, top, bottom and on a separate page, respectively
% \centering % Centers the table on the page, comment out to left-justify
% \begin{tabular}{l c c c c c} % The final bracket specifies the number of columns in the table along with left and right borders which are specified using vertical bars (|); each column can be left, right or center-justified using l, r or c. To specify a precise width, use p{width}, e.g. p{5cm}
% \toprule % Top horizontal line
% & \multicolumn{5}{c}{Growth Media} \\ % Amalgamating several columns into one cell is done using the \multicolumn command as seen on this line
% \cmidrule(l){2-6} % Horizontal line spanning less than the full width of the table - you can add (r) or (l) just before the opening curly bracket to shorten the rule on the left or right side
% Strain & 1 & 2 & 3 & 4 & 5\\ % Column names row
% \midrule % In-table horizontal line
% GDS1002 & 0.962 & 0.821 & 0.356 & 0.682 & 0.801\\ % Content row 1
% NWN652 & 0.981 & 0.891 & 0.527 & 0.574 & 0.984\\ % Content row 2
% PPD234 & 0.915 & 0.936 & 0.491 & 0.276 & 0.965\\ % Content row 3
% JSB126 & 0.828 & 0.827 & 0.528 & 0.518 & 0.926\\ % Content row 4
% JSB724 & 0.916 & 0.933 & 0.482 & 0.644 & 0.937\\ % Content row 5
% \midrule % In-table horizontal line
% \midrule % In-table horizontal line
% Average Rate & 0.920 & 0.882 & 0.477 & 0.539 & 0.923\\ % Summary/total row
% \bottomrule % Bottom horizontal line
% \end{tabular}
% \caption{Table caption text} % Table caption, can be commented out if no caption is required
% \label{tab:template} % A label for referencing this table elsewhere, references are used in text as \ref{label}
% \end{table}

%----------------------------------------------------------------------------------------

\mainmatter

%----------------------------------------------------------------------------------------
%	CHAPTER 1
%----------------------------------------------------------------------------------------

%\chapter{Chapter 1 Title}
%\label{ch:1}

%------------------------------------------------

% \section{Section 1 - Fullwidth Environment Example}

% \begin{fullwidth}
% \lipsum[5]
% \end{fullwidth}

% \subsection{Subsection 1}

% \lipsum[6-7]

% \subsection{Subsection 2}

% \lipsum[7-8]



%----------------------------------------------------------------------------------------

\backmatter

%----------------------------------------------------------------------------------------
%	BIBLIOGRAPHY
%----------------------------------------------------------------------------------------

\bibliography{bibliography} % Use the bibliography.bib file for the bibliography
\bibliographystyle{plainnat} % Use the plainnat style of referencing

%----------------------------------------------------------------------------------------

\printindex % Print the index at the very end of the document

\end{document}